\documentclass{lmcs}
\pdfoutput=1

\usepackage{enumerate}
\usepackage[colorlinks=true]{hyperref}
\usepackage{amssymb}
\usepackage{xcolor,latexsym,amsmath,extarrows,alltt}
\usepackage{xspace}
\usepackage{booktabs}
\usepackage{mathtools}
\usepackage{enumitem}
\usepackage{stmaryrd}
\usepackage{microtype}

\newcounter{dummy} \numberwithin{dummy}{section}
\theoremstyle{theorem}\newtheorem{theorem}[dummy]{Theorem}
\theoremstyle{theorem}\newtheorem{lemma}[dummy]{Lemma}
\theoremstyle{theorem}\newtheorem{corollary}[dummy]{Corollary}
\theoremstyle{definition}\newtheorem{definition}[dummy]{Definition}
\theoremstyle{definition}\newtheorem{example}[dummy]{Example}

\newcommand{\N}{\mathbb{N}}
\newcommand{\F}{\mathcal{F}}
\newcommand{\V}{\mathcal{V}}
\newcommand{\Vfree}{\mathcal{V}_{\mathit{free}}}
\newcommand{\Vbound}{\mathcal{V}_{\mathit{bound}}}
\newcommand{\Sorts}{\mathcal{S}}
\newcommand{\Types}{\mathcal{Y}}
\newcommand{\Terms}{\mathcal{T}}
\newcommand{\ATerms}{\mathcal{T}_{\mathcal{A}}}
\newcommand{\FOTerms}{\mathcal{T}_{\mathcal{FO}}}
\newcommand{\Rules}{\mathcal{R}}
\newcommand{\FV}{\mathit{FV}}
\newcommand{\BV}{\mathit{BV}}
\newcommand{\Positions}{\mathit{Positions}}
\newcommand{\Pairs}{\mathit{Pairs}}

\newcommand{\domain}{\mathtt{dom}}
\newcommand{\order}{\mathit{order}}

\newcommand{\asort}{\iota}
\newcommand{\bsort}{\kappa}
\newcommand{\atype}{\sigma}
\newcommand{\btype}{\tau}
\newcommand{\ctype}{\pi}
\newcommand{\dtype}{\alpha}
\newcommand{\identifier}[1]{\mathtt{#1}}
\newcommand{\afun}{\identifier{f}}
\newcommand{\bfun}{\identifier{g}}
\newcommand{\cfun}{\identifier{h}}
\newcommand{\avar}{x}
\newcommand{\bvar}{y}
\newcommand{\cvar}{z}
\newcommand{\Avar}{X}
\newcommand{\Bvar}{Y}
\newcommand{\Cvar}{Z}
\newcommand{\AFvar}{F}
\newcommand{\BFvar}{G}
\newcommand{\CFvar}{H}

\newcommand{\abs}[2]{\lambda #1.#2}

\newcommand{\arity}{\mathit{arity}}
\newcommand{\head}{\mathsf{head}}
\newcommand{\arrtype}{\rightarrow}
\newcommand{\arrz}{\Rightarrow}
\newcommand{\arr}[1]{\arrz_{#1}}
\newcommand{\arrr}[1]{\arr{#1}^*}
\newcommand{\subtermeq}{\unlhd}
\newcommand{\headsubtermeq}{\unlhd_{\bullet}}
\newcommand{\supterm}{\rhd}
\newcommand{\suptermeq}{\unrhd}

\newcommand{\symb}[1]{\mathtt{#1}}

\newcommand{\nul}{\symb{0}}
\newcommand{\one}{\symb{1}}
\newcommand{\nil}{\symb{nil}}
\newcommand{\cons}{\symb{cons}}
\newcommand{\strue}{\symb{true}}
\newcommand{\sfalse}{\symb{false}}
\newcommand{\suc}{\symb{s}}
\newcommand{\map}{\symb{map}}
\newcommand{\bool}{\symb{bool}}
\newcommand{\nat}{\symb{nat}}
\newcommand{\lijst}{\symb{list}}
\newcommand{\unitsort}{\mathtt{o}}

\newcommand{\cora}{\textsf{CORA}\xspace}

\newcommand{\secshort}{\S}
\newcommand{\myparagraph}[1]{\paragraph{\textbf{#1}}}

\setlength{\parindent}{0pt}
\setlength{\parskip}{\bigskipamount}
\setlist[itemize]{topsep=-\bigskipamount}

\newcommand{\mysubsection}[1]{\vspace{-12pt}\subsubsection{#1}}

\begin{document}

\title{COnstrained Rewriting Analyser: formalism}
\author{Cynthia Kop}
\address{Department of Software Science, Radboud University Nijmegen}
\email{C.Kop@cs.ru.nl}

\maketitle

\begin{abstract}
\cora\ is a tool meant to analyse constrained term rewriting systems, both
first-order and higher-order.  This document explains the underlying formalism.
\end{abstract}

\section{Types}

We fix a set $\Sorts$ of \emph{sorts} and define the set $\Types$ of \emph{types} inductively:
\begin{itemize}
\item all elements of $\Sorts$ are types (also called \emph{base types});
\item if $\atype,\btype \in \Types$ then $\atype \arrtype \btype$ is also a type (called an arrow
  type).
\end{itemize}
The arrow operator $\arrtype$ is right-associative, so all types can be denoted in a form
$\atype_1 \arrtype \dots \arrtype \atype_m \arrtype \asort$ with $\asort \in \Sorts$; we say the
\emph{arity} of this type is $m$, and the \emph{output sort} is $\asort$.

The \emph{order} of a type is recursively defined as follows:
\begin{itemize}
\item for $\asort \in \Sorts$: $\order(\asort) = 0$;
\item for arrow types: $\order(\atype \arrtype \btype) = \max(\order(\atype) + 1,\order(\btype))$.
\end{itemize}

\bigskip
Type equality is literal equality (i.e., $\atype_1 \arrtype \btype_1$ is equal to $\atype_2 \arrtype
\btype_2$ iff $\atype_1 = \atype_2$ and $\btype_1 = \btype_2$).

\subsection*{Remarks}

We do not impose limitations on the set of sorts.  In traditional, unsorted term rewriting, there
is only one sort (e.g., $\Sorts = \{ \unitsort \}$). However, we may also have a larger finite or
even infinite sort set.
In the future, we may consider a shallow form of polymorphic types, but for the moment we will limit
interest to these simple types.

\section{Unconstrained higher-order term rewriting systems (HOTRSs)}

Let us start by explaining systems without constraints. Most of the notions will be directly
relevant to constrained systems as well.

\subsection{Terms}
Terms are \emph{well-typed} expressions built over given sets of \emph{function symbols} and
\emph{variables}. The full definition is presented below.

\mysubsection{Symbols and variables}

We fix a set $\F$ of \emph{function symbols}, also called the \emph{alphabet}; each function symbol
is a \emph{typed constant}. Notation: $\afun \in \F$ or $(\afun : \atype) \in \F$ if we wish to
explicitly refer to the type (but the type should be considered implicit in the symbol).
Function symbols will generally be referred to as $\afun,\bfun,\cfun$ or using more suggestive
notation.

We also fix a set $\V$ of \emph{variables}, which are typed constants in the same way.  $\V$ should
be disjoint from $\F$, and we assume that $\V = \Vfree \uplus \Vbound$, where $\Vbound$ contains
infinitely many variables of each type; there are no restrictions on $\Vfree$.
Variables will generally be referred to as $\avar,\bvar,\cvar,\Avar,\Bvar,\Cvar,\AFvar,\BFvar,
\CFvar$ or using more suggestive notation.

\mysubsection{Term formation}

Terms are those expressions $s$ such that $s : \atype$ can be derived for some $\atype \in \Types$
using the following clauses:
\begin{itemize}
\item if $(\afun : \atype_1 \arrtype \dots \arrtype \atype_n \arrtype \btype) \in \F$ and
  $s_1 : \atype_1,\dots,s_n : \atype_n$ then $\afun(s_1,\dots,s_n) : \btype$;
\item if $(\avar : \atype_1 \arrtype \dots \arrtype \atype_n \arrtype \btype) \in \V$ and
  $s_1 : \atype_1,\dots,s_n : \atype_n$ then $\avar(s_1,\dots,s_n) : \btype$;
\item if $(\avar : \atype) \in \Vbound$ and $s : \btype$ then $\abs{\avar}{s} : \atype \arrtype
  \btype$.
\end{itemize}
A term of the form $\afun(s_1,\dots,s_n)$ is called a \emph{functional term} and $\afun$ is its
root.
A term of the form $\avar(s_1,\dots,s_n)$ is called a \emph{var term}, and $\avar$ is its variable.
A term of the form $\abs{\avar}{s}$ is called an \emph{abstraction} and $\avar$ is its variable.
If $s : \atype$ then we say that $\atype$ is the type of $s$; it is clear from the definitions
above that each term has a unique type.

Note that in the definition above, $n$ is not required to be maximal; for example, if
$\symb{greater} : \mathtt{int} \arrtype \mathtt{int} \arrtype \mathtt{bool}$, then each of
$\symb{greater}(),\symb{greater}(\avar)$ and $\symb{greater}(\avar,\bvar)$ are terms (with
distinct types). When no arguments are given (i.e., a term $\afun()$ or $\avar()$), we may omit
the brackets and just denote $\afun$ or $\avar$ for the term.  In this sense, the elements of $\F$
and $\V$ may all be considered terms.
A term $\avar$ can simply be called a variable (but is also still a var term);
a term $\afun$ may be called a constant (but is also still a functional term).

\mysubsection{$\alpha$-equality}
We let $=_\alpha$ be the usual $\alpha$-renaming equivalence relation as used in the
$\lambda$-calculus. This relation can be formally defined as follows:
\begin{itemize}
\item Let $\mu_0,\xi_0 : \V \rightarrow \N$ be defined as follows:
  $\mu_0(\avar) = 0$ and $\xi_0(\avar) = 0$ for all $\avar \in \V$.
\item Let $s =_\alpha t$ iff $s =_\alpha^{\mu_0,\xi_0,1} t$.
\item For $\mu,\xi : \V \rightarrow \N$ and $k \in \N$, let $=_\alpha^{\mu,\xi,k}$ be defined as
  follows:
  \begin{itemize}
  \item $\afun(s_1,\dots,s_n) =_\alpha^{\mu,\xi,k} t$ iff $t = \afun(t_1,\dots,t_n)$ with $s_1
    =_\alpha^{\mu,\xi,k} t_1,\dots,s_n =_\alpha^{\mu,\xi,k} t_n$;
  \item $\avar(s_1,\dots,s_n) =_\alpha^{\mu,\xi,k} t$ iff:
    \begin{itemize}
    \item $t = \bvar(t_1,\dots,t_n)$ and
    \item $s_1 =_\alpha^{\mu,\xi,k} t_1,\dots,s_n =_\alpha^{\mu,\xi,k} t_n$, and
    \item either $\avar = \bvar$ and $\mu(\avar) = \xi(\avar) = 0$,
      or $\mu(\avar) = \xi(\avar) > 0$.
    \end{itemize}
  \item $\abs{\avar}{s} =_\alpha^{\mu,\xi,k} t$ iff
    $t = \abs{\bvar}{u}$ and $s =_\alpha^{\mu[\avar:=k],\xi[\bvar:=k],k+1} u$. \\
    (Here, $\mu[\avar:=k]$ is the function that maps $\avar$ to $k$ and all other $\cvar$ to
    $\mu(\cvar)$; similar for $\xi[\bvar:=k]$.)
  \end{itemize}
\end{itemize}
That is, we progressively descend into the term and keep track of where variables are bound; the
structure of the two terms has to be exactly the same, and function symbols and unbound variables
should occur at the same positions in both terms. However, when encountering a bound variable, we
only require that this variable was bound by the same $\lambda$ in both terms.
We can straightforwardly prove that $=_\alpha$ is an equivalence relation (Corollary
\ref{corr:alphaequiv}).

%\mysubsection{$\eta$-equality}
%
%We let $=_\eta$ be the equivalence relation on terms generated by: for $a \in \F \cup \V$:
%$\abs{x}{a(s_1,\dots,s_n,x)} =_\eta a(s_1,\dots,s_n)$ provided $x \in \Vbound$ does not occur
%in $s_1,\dots,s_n$.

\mysubsection{Sets of terms}

We denote $\Terms(\F,\V)$ for the set of all terms $s$, modulo $=_\alpha$.  In practice, we will
reason with terms rather than equivalence classes, but always consider equality modulo $=_\alpha$.

A term $s$ is a \emph{pattern} if for every subterm $t \subtermeq s$ we have: if $t$ is a var term
$x(s_1,\dots,s_n)$ with $x \in \Vfree$, then $s_1,\dots,s_n$ are distinct elements of $\Vbound$.
Patterns will be relevant in rule formation for specific limitations of HOTRSs.

A term $s$ is \emph{applicative} if it does not use variables in $\Vbound$ (and therefore, no
abstractions are constructed); that is, a term is applicative if it can be typed by the
following clauses:
\begin{itemize}
\item if $(\afun : \atype_1 \arrtype \dots \arrtype \atype_n \arrtype \btype) \in \F$ and
  $s_1 : \atype_1,\dots,s_n : \atype_n$ then $\afun(s_1,\dots,s_n) : \btype$;
\item if $(\avar : \atype_1 \arrtype \dots \arrtype \atype_n \arrtype \btype) \in \Vfree$ and
  $s_1 : \atype_1,\dots,s_n : \atype_n$ then $\avar(s_1,\dots,s_n) : \btype$.
\end{itemize}
The set of applicative terms is denoted $\ATerms(\F,\V)$.  Note that $=_\alpha$ is the
identity on applicative terms, so $\ATerms(\F,\V) \subsetneq \Terms(\F,\V)$.
Note also that, in an applicative \emph{pattern}, variables are not allowed to occur at the head
at all: every var term is a variable.

A term is \emph{first-order} if its type can be derived by the following clauses:
\begin{itemize}
\item if $(\afun : \atype_1 \arrtype \dots \arrtype \atype_n \arrtype \asort) \in \F$ with $\asort
  \in \Sorts$ and
  $s_1 : \atype_1,\dots,s_n : \atype_n$ then $\afun(s_1,\dots,s_n) : \asort$;
\item if $(\avar : \asort) \in \Vfree$ and $\asort \in \Sorts$, then $\avar : \asort$.
\end{itemize}
The set of first-order terms is denoted $\FOTerms(\F,\V)$.  Note that every first-order term is
also an applicative term (indeed -- every first-order term is an applicative \emph{pattern}),
so $\FOTerms(\F,\V) \subsetneq \ATerms(\F,\V) \subsetneq \Terms(\F,\V)$.

\subsection{Term variables}
The set of \emph{free variables} of a term is inductively defined as follows:
\begin{itemize}
\item $\FV(\afun(s_1,\dots,s_n)) = \FV(s_1) \cup \dots \cup \FV(s_n)$;
\item $\FV(\avar(s_1,\dots,s_n)) = \{ \avar \} \cup \FV(s_1) \cup \dots \cup \FV(s_n)$;
\item $\FV(\abs{\avar}{s}) = \FV(s) \setminus \{ \avar \}$.
\end{itemize}
That is, $\FV(s)$ contains all variables in $s$ except for those bound by a $\lambda$.
For applicative and first-order terms $s$, this is the set of \emph{all} variables occurring in
$s$.  A term $s$ is \emph{closed} if $\FV(s) = \emptyset$.

Since $\FV(s) = \FV(t)$ whenever $s =_\alpha t$ (Lemma \ref{corr:alphafreevar}),
$\FV$ also defines a function on equivalence classes of terms.

The set of \emph{bound variables} of a term is inductively defined as follows:
\begin{itemize}
\item $\BV(\afun(s_1,\dots,s_n)) = \BV(s_1) \cup \dots \cup \BV(s_n)$;
\item $\BV(\avar(s_1,\dots,s_n)) = \BV(s_1) \cup \dots \cup \BV(s_n)$;
\item $\BV(\abs{\avar}{s}) = \BV(s) \cup \{ \avar \}$.
\end{itemize}
That is, $\BV(s)$ contains all variables that are bound by a $\lambda$ in $s$.  Note that all
elements in $\BV(s)$ are in $\Vbound$.
Clearly, $\BV$ does \emph{not} define a function on equivalence classes.

\subsection{Subterms and positions}

The \emph{positions} of a given term are the paths to specific subterms, defined as follows:

\begin{itemize}
\item $\Positions(\afun(s_1,\dots,s_n)) = \{ \epsilon \} \cup \{ i \cdot p \mid 1 \leq i
  \leq n \wedge p \in \mathit{Positions}(s_i) \}$;
\item $\Positions(\avar(s_1,\dots,s_n)) = \{ \epsilon \} \cup \{ i \cdot p \mid 1 \leq i
  \leq n \wedge p \in \Positions(s_i) \}$;
\item $\Positions(\abs{\avar}{s}) = \{ 1 \cdot p \mid p \in \Positions(s) \}$.
\end{itemize}
Note that positions are associated to a term; thus, not every sequence of natural numbers is a
position.

For a term $s$ and a position $p \in \Positions(s)$, the \emph{subterm of $s$ at position $p$},
denoted $s|_p$, is defined as follows:
\begin{itemize}
\item $s|_\epsilon = s$;
\item $\afun(s_1,\dots,s_n)|_{i \cdot p} = s_i|_p$;
\item $\avar(s_1,\dots,s_n)|_{i \cdot p} = s_i|_p$;
\item $(\abs{\avar}{s})|_{1 \cdot p} = s|_p$.
\end{itemize}

If $s|_p$ has the same type as some term $t$, then $s[t]_p$ denotes $s$ with the subterm at position
$p$ replaced by $t$.  Formally, $s[t]_p$ is obtained as follows:
\begin{itemize}
\item $s[t]_\epsilon = t$;
\item $\afun(s_1,\dots,s_n)[t]_{i \cdot p} = \afun(s_1,\dots,s_{i-1},s_i[t]_p,s_{i+1},\dots,s_n)$;
\item $\avar(s_1,\dots,s_n)[t]_{i \cdot p} = \avar(s_1,\dots,s_{i-1},s_i[t]_p,s_{i+1},\dots,s_n)$.
\end{itemize}
Thus, we can find and replace the subterm at a given position.

We say that \emph{$t$ is a subterm of $s$}, notation $t \subtermeq s$, if there is some position
$p \in \Positions(s)$ with $t = s|_p$.  This could equivalently be formulated as follows:

\begin{lemma}
$t \subtermeq s$ if and only if one of the following holds:
\begin{itemize}
\item $s = t$;
\item $s = \afun(s_1,\dots,s_n)$ or $s = \avar(s_1,\dots,s_n)$ and $t \subtermeq s_i$ for some $i$;
\item $s = \abs{x}{s'}$ and $t \subtermeq s'$.
\end{itemize}
\end{lemma}

We also observe that $\subtermeq$ is transitive:

\begin{lemma}
If $s \subtermeq t$ and $t \subtermeq q$ then $s \subtermeq q$.
\end{lemma}

This is obvious because if $t = q|_p$ and $s = t|_{p'}$ then $s = q|_{p \cdot p'}$.

It should be noted that in contrast to most definitions of higher-order rewriting, we do \emph{not}
consider, for example, $\afun(x)$ to be a subterm of $\afun(x,y)$.  Instead, we define the
following: \emph{$t$ is a head-subterm of $s$}, notation $t \headsubtermeq s$ if one of the
following holds:
\begin{itemize}
\item $t \subtermeq s$;
\item $t = \afun(s_1,\dots,s_i)$ and there exist $s_{i+1},\dots,s_n$ such that
  $\afun(s_1,\dots,s_n) \subtermeq s$;
\item $t = \avar(s_1,\dots,s_i)$ and there exist $s_{i+1},\dots,s_n$ such that
  $\avar(s_1,\dots,s_n) \subtermeq s$;
\end{itemize}
So, the head-subterms of $s$ are both the subterms of $s$, and those terms that occur as the head
of a subterm of $s$.

It should also be noted that if $s =_\alpha t$, it does not follow that $s$ and $t$ have the same
subterms: $\abs{x}{x}$ has a subterm $x$, while $\abs{y}{y}$ does not; however, subterms are the
same modulo renaming of variables in $\Vbound$.  For applicative and
first-order terms, this is not an issue since there each equivalence class contains only one
element.

Regarding different kinds of terms: the subterms and positions of a first-order term by these
definitions are exactly the subterms and positions as they are usually considered in first-order
term rewriting; however, head-subterms are generally not considered.  For applicative terms,
both subterms and head-subterms are usually referred to as just ``subterms''; we distinguish them
here because doing so is practical for analysis.

\subsection{Application and substitution}

A \emph{substitution} is a partial function $\gamma$ that maps variables $\avar \in \V$ to a term
$\gamma(\avar)$ of the same type.  The \emph{domain} of this partial function is denoted
$\domain(\gamma)$.
We denote $[\avar_1:=s_1,\dots,\avar_n:=s_n]$ for the substitution $\gamma$ on domain $\{\avar_1,
\dots,\avar_n\}$ with $\gamma(x_i) = s_i$ for $1 \leq i \leq n$.

Applying a substitution $\gamma$ to a term $s$, notation $s\gamma$, yields a new term of the same
type, as we will define below. However, this requires a separate definition of \emph{application}.
We will define the notions of term application and substitution in a mutually recursive manner.

\mysubsection{Term application}\label{mysubsec:application}
A term $s : \atype_1 \arrtype \dots \arrtype \atype_m \arrtype \btype$ can be applied to a sequence
$[t_1,\dots,t_m]$ of terms, provided $t_1 : \atype_1,\dots,t_m : \atype_m$, through the following
clauses:
\begin{itemize}
\item $s \cdot [t_1,\dots,t_m] = s$ if $m = 0$;
\item $s \cdot [t_1,\dots,t_m] = (s \cdot t_1) \cdot [t_2,\dots,t_m]$ otherwise;
\item if $s = \afun(s_1,\dots,s_n)$ then $s \cdot t = \afun(s_1,\dots,s_n,t)$;
\item if $s = \avar(s_1,\dots,s_n)$ then $s \cdot t = \avar(s_1,\dots,s_n,t)$;
\item if $s = \abs{\avar}{s}$ then $s \cdot t = s[\avar:=t]$ (using substitution;
  see section \ref{mysubsec:substitution}).
\end{itemize}

\bigskip
Note that for applicative terms, this definition is complete as substitution is only needed for
abstractions.  We have the following result;

\begin{lemma}\label{lem:applicative_notation}
The set $\ATerms(\F,\V)$ is the smallest set such that:
\begin{itemize}
\item $\F \cup \V \subseteq \ATerms(\F,\V)$;
\item if $s,t \in \ATerms(\F,\V)$ and $s : \atype \arrtype \btype$ and $t : \atype$ then
  $s \cdot t \in \ATerms(\F,\V)$.
\end{itemize}
\end{lemma}

\begin{proof}
Trivial.
\end{proof}

Lemma~\ref{lem:applicative_notation} shows that our applicative terms are the same as applicative
terms constructed in the traditional way; however, for convenience we denote them in a functional
notation.

\mysubsection{Substitution}\label{mysubsec:substitution}
A substitution $\gamma$ can be applied to a term as follows:
\begin{itemize}
\item $\afun(s_1,\dots,s_n)\gamma = \afun(s_1\gamma,\dots,s_n\gamma)$;
\item $\avar(s_1,\dots,s_n)\gamma = \avar(s_1\gamma,\dots,s_n\gamma)$
  if $\avar \notin \domain(\gamma)$;
\item $\avar(s_1,\dots,s_n)\gamma = \gamma(\avar) \cdot [s_1\gamma,\dots,s_n\gamma]$
  if $\avar \in \domain(\gamma)$ \\
  (using application; see section \ref{mysubsec:application});
\item $(\abs{\avar}{s})\gamma = \abs{\cvar}{(s ([\avar:=\cvar] \cup [\bvar := \gamma(\bvar) \mid
  \bvar \in \domain(\gamma) \setminus \{\avar\}])})$ \\
  for $\cvar$ a \emph{fresh}** variable in $\Vbound$ with the same type as $\avar$.
\end{itemize}
** A \emph{fresh} variable $\cvar$ is one that does not occur in $\FV(\bvar\gamma)$ for any
$\bvar \in \FV(\abs{\avar}{s})$.

Due to the mutual recursion, it is not obvious that these definitions are well-founded: it is not
the case that each step is defined in terms of an obviously smaller substitution / application.
However, by an induction on types we can see that this is well-defined: for all $s,\gamma$ there
exists $t$ such that $s\gamma = t$; and for all appropriately typed $s_0,s_1,\dots,s_m$ there
exists $s$ such that $s = s_0 \cdot [s_1,\dots,s_m]$ (Lemma \ref{lem:substdefined}).

More critically, these definitions technically do not define functions on terms: the substitution
of an abstraction may lead to any fresh variable being chosen.  Hence, we should perhaps think of
these notions as defining \emph{relations} between terms.  However, they do define a function on
\emph{equivalence classes} (by Corollary \ref{cor:substitutionalpha}):
\begin{itemize}
\item if $s_1 =_\alpha s_2$ and $s_1\gamma = t_1$ and $s_2\gamma' = t_2$ and $\gamma(x) =_\alpha
  \gamma'(x)$ for $x \in \FV(s_1)$, then $t_1 =_\alpha t_2$;
\item if $s_1 =_\alpha s_2$ and $s_1\gamma = t_1$ and $s_2\gamma = t_2$ then $t_1 =_\alpha t_2$
  (this is a special case of the previous item, but it is highlighted since it shows that
  substitution by a specific $\gamma$ defines a function);
\item if $s_0 =_\alpha t_0$, \dots, $s_m =_\alpha t_m$, and $s_0 \cdot [s_1,\dots,s_m] = s$ and
  $t_0 \cdot [t_1,\dots,t_m] = t$, then $s =_\alpha t$.
\end{itemize}
Hence, the difference is not significant, and we can safely think of substitution and application
as defining functions.

A \emph{renaming} is a substitution $[x_1:=y_1,\dots,x_n:=y_n]$ with $x_1,\dots,x_n$ pairwise
distinct, and $y_1,\dots,y_n$ pairwise distinct.

========================

For two substitutions $\gamma$ and $\delta$, we let $\gamma\delta$ denote the substitution
$[\avar := \gamma(\avar)\delta \mid \avar \in \domain(\gamma)] \cup
[\avar := \delta(\avar) \mid \avar \in \domain(\delta) \setminus \domain(\gamma)]$.
Essentially, applying a substitution $\gamma$ to a term corresponds with replacing each variable
$\avar$ by $\avar\gamma$ (and evaluating the applications of abstractions that are created as a
result), and applying $\gamma\delta$ corresponds to replacing $\avar$ by $(\avar\gamma)\delta$.
We will see in Lemma \ref{lem:combinesubst} that $s(\gamma\delta)$ is exactly $(s\gamma)\delta$.

\subsection{Some results on $\alpha$-equivalence and substitution}

We have seen that substitution and application are well-defined.  In this section, we will prove
that they in fact define a function on $=_\alpha$-equivalence classes.  In addition, we will obtain
several more results that will make it easier to reason modulo $\alpha$.

\subsubsection{Helper results}
We start with some helper results (which will be used in the proofs of other results, but never
quietly assumed known):

\newcommand{\aprel}[1]{\mathsf{ap}(#1)}
\newcommand{\subrel}[1]{\mathsf{subst}(#1)}


Focusing on application, we obtain the following lemma:

\begin{lemma}\label{lem:apadd}
If $\aprel{s_0,[s_1,\dots,s_n],s}$ and $\aprel{s,t,u}$ then $\aprel{s_0,[s_1,\dots,s_n,t],u}$.
\end{lemma}

\begin{proof}
By induction on $n$.
If $n = 0$, then $s_0 = s$ so $\aprel{s_0,[t],u}$ is given.
If $n = 1$ then $\aprel{s_0,[s_1,t],u}$ holds because $\aprel{s_0,s_1,s}$ and $\aprel{s,t,u}$.
If $n > 1$ then $\aprel{s_0,[s_1,\dots,s_n],s}$ holds because there is $q$ such that
$\aprel{s_0,s_1,q}$ and $\aprel{q,[s_2,\dots,s_n],s}$. But then by the induction hypothesis,
$\aprel{q,[s_2,\dots,s_n,t],u}$.
\end{proof}

We also obtain the following technical lemma, which will be very useful to see that combining
substitutions functions in the expected way.

==================================

TODO: ik vermoed dat deze lemmas niet zomaar gaan volgen! Het is beter om uit te vogelen wat ik
echt nodig heb, en dan alpha-equality te gebruiken!


\begin{lemma}\label{lem:applicationsubstitution}
We have:
\begin{enumerate}
\item\label{lem:applicationsubstitution:ap} Let $s,s_0,\dots,s_m$ and $t,t_0,\dots,t_m$ be terms,
  and $\gamma$ a substitution.  Suppose that $\aprel{s_0,[s_1,\dots,s_m],s}$ and $\aprel{t_0,
  [t_1,\dots,t_m],t}$, and that for all $i \in \{0,\dots,m\}$ we have
  $\subrel{s_i,\gamma,t_i}$.
  Then $\subrel{s,\gamma,t}$.
\item\label{lem:applicationsubstitution:apex} Let $s,s_0,\dots,s_m$ and $t$ be terms, and
  $\gamma$ a substitution.  If $\aprel{s_0,[s_1,\dots,s_m],s}$ and $\subrel{s,\gamma,t}$,
  then there exist $t_0,\dots,t_m$ such that $\subrel{s_i,\gamma,t_i}$ for all $i$, and
  $\aprel{t_0,[t_1,\dots,t_m],t}$.
\item\label{lem:applicationsubstitution:sub}
  Let $\gamma$ be a substitution, and $u,u^\gamma$ such that $\subrel{u,\gamma,u^\gamma}$.
  Let $x,z$ be variables, and let $\delta$ denote the substitution $[x:=z] \cup [y:=\gamma(y)
  \mid y \in \domain(\gamma) \setminus \{x\}]$.
  Let $q,q^\delta$ be terms such that:
  \begin{itemize}
  \item $z \notin \FV(q) \setminus \{x\}$
  \item $z \notin \FV(\gamma(y))$ for any $y \in \domain(\gamma) \cap (\FV(q) \setminus \{x\})$
  \item $\subrel{q,\delta,q^\delta}$. \\
  \end{itemize}
  Let $s,t$ be terms such that $\subrel{q,[x:=u],s}$ and $\subrel{q^\delta,[z:=u^\gamma],t}$.
  Then $\subrel{s,\gamma,t}$.
\end{enumerate}
\end{lemma}

\begin{proof}
By a mutual induction on the definitions of $\aprel{s_0,[s_1,\dots,s_m],s}$ and
$\subrel{q,[x:=u],s}$ respectively.

For (\ref{lem:applicationsubstitution:ap}), consider $m$ and the form of $s_0$.
\begin{itemize}
\item If $m = 0$, then $s_0 = s$, $t_0 = t$ and $\subrel{s,\gamma,t}$ is given.
\item If $m > 1$, then by definition there exist $q$ with $\aprel{s_0,s_1,q}$ and
  $\aprel{q,[s_2,\dots,s_m],s}$, and $w$ with $\aprel{t_0,t_1,w}$ and $\aprel{w,
  [t_2,\dots,t_m],t}$. By the induction hypothesis on $\aprel{s_0,s_1,q}$, we have
  $\subrel{q,\gamma,w}$.  By the induction hypothesis on $\aprel{q,[s_2,\dots,s_m],s}$ we
  then have $\subrel{s,\gamma,w}$.
\item If $m = 1$ and $s_0 = f(q_1,\dots,q_n)$, then we can write $t_0 = f(w_1,\dots,w_n)$
  with $\subrel{q_i,\gamma,w_i}$ for $1 \leq i \leq n$. Then $s = f(q_1,\dots,q_n,s_1)$ and
  $t = f(w_1,\dots,w_n,t_1)$.  We immediately obtain $\subrel{s,\gamma,t}$.
\item If $m = 1$ and $s_0 = x(q_1,\dots,q_n)$ with $x \notin \domain(\gamma)$, then we complete
  essentially as above.
\item If $m = 1$ and $s_0 = x(q_1,\dots,q_n)$ with $x \in \domain(\gamma)$, then $s = x(q_1,
  \dots,q_n,s_1)$ and we have $\aprel{\gamma(x),[w_1,\dots,w_n],t_0}$ for some $\vec{w}$ such
  that $\subrel{q_i,\gamma,w_i}$ for all $i$. By Lemma \ref{lem:apadd}, we have
  $\aprel{\gamma(x),[w_1,\dots,w_n,t_1],t}$. But then $\subrel{s,\gamma,t}$ holds.
\item If $m = 1$ and $s_0 = \abs{x}{q}$, then $\subrel{q,[x:=s_1],s}$.  Moreover, $t_0 =
  \abs{z}{w}$ for some $z \notin \FV(\abs{x}{q}) \cup \bigcup_{y \in \FV(\abs{x}{q}) \cap
  \domain(\gamma)} \FV(\gamma(y))$ and $w$ with $\subrel{q,[x:=z] \cup [y:=\gamma(y) \mid
  y \in \domain(\gamma) \setminus \{x\}], w}$, and $\subrel{w,[z:=t_1],t}$.
  We apply (\ref{lem:applicationsubstitution:sub}) of the induction hypothesis (on the
  derivation $\subrel{q,[x:=s_1],s}$) to obtain $\subrel{s,\gamma,t}$ as required.
\end{itemize}

For (\ref{lem:applicationsubstitution:apex}), we also consider $m$ and the form of $s_0$.
\begin{itemize}
\item If $m = 0$, then $s = s_0$, so we are done choosing $t_0 := t$.
\item If $m > 1$, then by definition there exists $q$ with $\aprel{s_0,s_1,q}$ and
  $\aprel{q,[s_2,\dots,s_m],s}$. By the induction hypothesis on $\aprel{q,[s_2,\dots,s_m],s}$
  there exist $w,t_2,\dots,t_m$ such that $\aprel{w,[t_2,\dots,t_m],t}$ and $\subrel{q,\gamma,w}$
  and $\subrel{s_i,\gamma,t_i}$ for all $i \geq 2$. By the induction hypothesis on
  $\aprel{s_0,s_1,q}$ there exist $t_0,t_1$ such that $\aprel{t_0,t_1,w}$ and $\subrel{s_i,
  \gamma,t_i}$ for $i \in \{0,1\}$.
\item If $m = 1$ and $s_0 = f(q_1,\dots,q_n)$ then $s = f(q_1,\dots,q_n,s_1)$.
  From $\subrel{s,\gamma,t}$ we obtain $w_1,\dots,w_n,t_1$ such that $t = f(w_1,\dots,s_n,t_1)$ and
  $\subrel{q_i,\gamma,w_i}$ for $1 \leq i \leq n$ and $\subrel{s_1,\gamma,t_1}$.
  Let $t_0 := f(w_1,\dots,s_n)$. Then indeed $\subrel{s_0,\gamma,t_0}$ and $\aprel{t_0,[t_1],t}$.
\item If $m = 1$ and $s_0 = x(q_1,\dots,q_n)$ with $x \notin \domain(\gamma)$ then $s = x(q_1,
  \dots,q_n,s_1)$.  We find $w_1,\dots,w_n,t_1$ with $t = x(w_1,\dots,w_n,t_1)$ as above, and let
  $t_0 := x(w_1,\dots,w_n)$.
\item If $m = 1$ and $s_0 = x(q_1,\dots,q_n)$ with $x \in \domain(\gamma)$ then $s = x(q_1,\dots,
  q_n,s_1)$ and $\aprel{\gamma(x),[w_1,\dots,w_n,t_1],t}$ with $w_1,\dots,w_n,t_1$ as above.
  Let $t_0$ be such that $\aprel{\gamma(x),[w_1,\dots,w_n],t_0}$, which exists by Lemma
  \ref{lem:substdefined}.  Then clearly $\subrel{s_0,\gamma,t_0}$, and $\aprel{t_0,t_1,t}$ follows
  by TODO
\item ===========================
\end{itemize}

For (\ref{lem:applicationsubstitution:sub}), consider the form of $q$.
\begin{itemize}
\item If $q = \afun(q_1,\dots,q_n)$ then necessarily $q^\delta = \afun(q_1^\delta,\dots,q_n^\delta)$
  with $\subrel{q_i,\delta,q_i^\delta}$ for all $i$.
  Then $\subrel{q,[x:=u],s}$ implies $s = \afun(s_1,\dots,s_n)$ with $\subrel{q_i,[x:=u],s_i}$ for
  all $i$, and $\subrel{q^\delta,[z:=u^\gamma],t}$ implies $t = \afun(t_1,\dots,t_n)$ with
  $\subrel{q_i^\delta,[z:=u^\gamma],t_i}$. By the induction hypothesis, $\subrel{s_i,\gamma,t_i}$
  holds for all $i$, which gives $\subrel{s,\gamma,t}$.
\item If $q = y(q_1,\dots,q_n)$ with $y \neq x$ and $y \notin \domain(\gamma)$, then also
  $y \notin \domain(\delta)$. Then we complete essentially as above, since also $q^\delta =
  y(q_1^\delta,\dots,q_n^\delta)$ and $s = y(s_1,\dots,s_n)$ and $t = y(t_1,\dots,t_n)$.
\item If $q = x(q_1,\dots,q_n)$ then $\aprel{z,[q_1^\delta,\dots,q_n^\delta],q^\delta}$ for some
  $q_1^\delta,\dots,q_n^\delta$ with $\subrel{q_i,\delta,q_i^\delta}$ for all $i$; that is,
  $q^\delta = z(q_1^\delta,\dots,q_n^\delta)$.
  From $\subrel{q,[x:=u],s}$ we obtain $\aprel{u,[s_1,\dots,s],s}$ where $\subrel{q_i,[x:=u],s_i}$
  for all $i$;
  and from $\subrel{q^\delta,[z:=u^\gamma],t}$ we obtain $\aprel{u^\gamma,[t_1,\dots,t_n],t}$ where
  $\subrel{q_i^\delta,[z:=u^\gamma],t_i}$ for all $i$.
  By the induction hypothesis, $\subrel{s_i,\gamma,t_i}$ for all $i$.
  By induction hypothesis (\ref{lem:applicationsubstitution:ap}) applied on
  $\aprel{u,[s_1,\dots,s_n],s}$ we obtain $\subrel{s,\gamma,t}$ as required.
\item If $q = y(q_1,\dots,q_n)$ with $y \in \domain(\gamma) \setminus \{x\}$, then we have
  $\aprel{y,[s_1,\dots,s_n],s}$, so $s = y(s_1,\dots,s_n)$, for some $s_1,\dots,s_n$ with
  $\subrel{q_i,[x:=u],s_i}$.  Moreover, we have $\aprel{\delta(y),[q_1^\delta,\dots,q_n^\delta],
  q^\delta}$ for some $q_1^\delta,\dots,q_n^\delta$ such that $\subrel{q_i,\delta,q_i^\delta}$
  for all $i$, and $\delta(y) = \gamma(y)$ in this case.
  From $\subrel{q^\delta,[z:=u^\gamma],t}$ we obtain TODO
\item ====================
\end{itemize}
\end{proof}

==================================

\begin{lemma}\label{lem:combinesubsthelper}
Let $\gamma,\delta,\eta$ be substitutions, such that $\domain(\eta) = \domain(\gamma) \cup \domain(\delta)$
and for all $x$ in this domain: if $x \in \domain(\gamma)$ then $\subrel{\gamma(x),\delta,\eta(x)}$;
otherwise $\eta(x) = \delta(x)$.  Let $s,t,u$ be terms.  Then:
\begin{enumerate}
\item\label{lem:combinesubsthelper:subst}
  If both $\subrel{s,\gamma,t}$ and $\subrel{t,\delta,u}$, then $\subrel{s,\eta,u}$.
\item\label{lem:combinesubsthelper:appl}
  If $\aprel{t_0,[t_1,\dots,t_n],t}$ and $\subrel{t_i,\delta,u_i}$ for all $i$ as well as
  $\subrel{t,\delta,u}$, then $\aprel{u_0,[u_1,\dots,u_n],u}$.
\end{enumerate}
\end{lemma}

Note that (\ref{lem:combinesubsthelper:subst}) exactly indicates that if $s\gamma = t$ and $t\delta = u$
then $s(\gamma\delta) = u$; put differently, that $(s\gamma)\delta = s(\gamma\delta)$.

\begin{proof}
By induction on the mutual definition of $\subrel{s,\gamma,t}$ and $\aprel{t_0,[t_1,\dots,t_m],t}$

For (\ref{lem:combinesubsthelper:subst}), consider the form of $s$.

\begin{itemize}
\item If $s = \afun(s_1,\dots,s_n)$ then $t = \afun(t_1,\dots,t_n)$ with $\subrel{s_i,\gamma,t_i}$;
  and $u = \afun(u_1,\dots,u_n)$ with $\subrel{t_i,\delta,u_i}$.  By the induction hypothesis,
  $\subrel{s_i,\eta,u_i}$ for all $i$.  Hence, $\subrel{s,\eta,u}$ follows by \ref{subst:func} in
  the definition of $\subrel{}$.
\item If $s = x(s_1,\dots,s_n)$ and $x \notin \domain(\gamma)$, then $t = x(t_1,\dots,t_n)$ with
  $\subrel{s_i,\gamma,t_i}$ for all $i$.  By definition of $\subrel{t,\delta,u}$ we have
  $\aprel{\delta(x),[u_1,\dots,u_n],u}$ for $u_1,\dots,t_n$ with $\subrel{t_i,\delta,u_i}$ for all
  $i$. By the induction hypothesis, $\subrel{s_i,\eta,u_i}$ for all $i$.  Also, in this case
  $\eta(x) = \delta(x)$.  Hence, indeed $\aprel{s,\eta,u}$ by \ref{subst:safevar} in the definition
  of $\subrel{}$.
\item If $s = x(s_1,\dots,s_n)$ and $x \in \domain(\gamma)$, then $\aprel{\gamma(x),[t_1,\dots,t_n],
  t}$ for $t_1,\dots,t_n$ with $\subrel{s_i,\gamma,t_i}$ for all $i$.
\item  ===========

then $\aprel{\gamma(x),[t_1,\dots,t_n],t}$ holds for some $t_1,
  \dots,t_n$ with $\subrel{s_i,\gamma,t_i}$ for all $i$.
  Let



  By the induction hypothesis on $\subrel{s_i,\gamma,t_i}$, we have $\subrel{s_i,\eta,u_i}$

  TODO.

  Now: $\subrel{t,\delta,u}$ is weird.

  I need: $\subrel{\eta(x),[u_1,\dots,u_n],u}$

\item If $s = \abs{x}{s'}$ then let $z,z'$ be variables which do not occur in $\FV(y\gamma)$ or
  $\FV(y\delta)$ or $\FV(y\eta)$ for any $y$ in their domain, nor in the domain of $\gamma$,
  $\delta$ or $\eta$.
  Let $\gamma^{x\mapsto z}$ denote the substitution $[x:=z] \cup [y := \gamma(y) \mid y \in
  \domain(\gamma) \setminus \{x\}]$, and similar for $\eta^{x \mapsto z'}$ and $\delta^{z
  \mapsto z'}$.
  Then by definition, $s\gamma = \abs{z}{(s'\gamma^{x\mapsto z})}$ and $s\eta =
  \abs{z'}{(s'\eta^{x\mapsto z'})}$.

  To prove the case, we must show that there exist $t'$ and $u'$ such that
  $s'\gamma^{x\mapsto z} = t'$ and $s'\eta^{x\mapsto z} = u'$ and
  $t'\delta^{z \mapsto z'} = u'$.
  If we have this, then we can complete by choosing $t := \abs{z}{t'}$ and $u := \abs{z'}{u'}$:
  this holds because $(\abs{z}{t'})\delta = \abs{z'}{(t'\delta^{z \mapsto z'})}$ since $z'$ does
  not occur in $\FV(y\delta)$ for any $y \in \FV(t')$ (since $z'$ does not occur in any
  $\delta(y)$, and $z' \notin \FV(t')$ since TODO -- $z'$ does not occur in any $\gamma(y)$ for
  $y \in \FV(s)$.

  This holds by the induction hypothesis if we can prove that
  $\eta^{x \mapsto z} = \gamma^{x\mapsto z}\delta^{z\mapsto z'}$.
  Then we indeed find $t',u'$ such that $s'\gamma^{x\mapsto z} = t'$ and $s'\eta^{x\mapsto z} = u'$
  and $t'\delta^{z \mapsto z'} = u'$, 

% ((λx.s')γ)δt = (λy'.s'([x:=y'] ∪ [z:=γ(z) | z ∈ dom(γ) ∧ z ≠ x]))δ for fresh y'
%              = λy.s([x:=y'] ∪ [z:=γ(z) | z ∈ dom(γ) ∧ z ≠ x])([y':=y] ∪ [z:=δ(z) | z ∈ dom(δ) ∧ z ≠ y']) for fresh y
%              = λy.s([x:=y'] ∪ [z:=γ(z) | z ∈ dom(γ) ∧ z ≠ x])([y':=y] ∪ δ) because y' is fresh, so not in dom(δ)
%              = λy.s( ([x:=y'] ∪ [z:=γ(z) | z ∈ dom(γ) ∧ z ≠ x])([y':=y] ∪ δ) ) by the IH
%              = λy.s( [x:=y] ∪ [z:=η(z) | z ∈ dom(γ) ∧ z ≠ x] ) by analysis of what happens with each variable
%              = (λy.s)η
%
% Let's do the important analysis! For variables z in the given domain:
% * if z = x, then ([x:=y'] ∪ [z:=γ(z) | z ∈ dom(γ) ∧ z ≠ x])([y':=y] ∪ δ)(z) = y'([y':=y] ∪ δ) = y
%   and ([x:=y] ∪ [z:=η(z) | z ∈ dom(γ) ∧ z ≠ x])(y) = y as well
% * if z ≠ x and not z ∈ dom(γ) then ([x:=y'] ∪ [z:=γ(z) | z ∈ dom(γ) ∧ z ≠ x])([y':=y] ∪ δ)(z) = z(y':=y] ∪ δ);
%   since z ≠ y' (as y' was fresh, so is not in dom(δ)), this is δ(z)
%   this is exactly η(z) as well
% * finally, if z ≠ x and z ∈ dom(γ) then ([x:=y'] ∪ [z:=γ(z) | z ∈ dom(γ) ∧ z ≠ x])([y':=y] ∪ δ)(z) = γ(z)([y':=y] ∪ δ)
%   we are done if we can prove that this is exactly γ(z)δ because y' does not occur in γ(z)
\end{itemize}
\end{proof}

\subsubsection{Results on substitution}
With the help of Lemma \ref{lem:substitutionalphahelper}, we easily obtain the important result that
states (among other things) that substitution defines a function on $\Terms(\F,\V)$.

\begin{lemma}\label{lem:substitutionalpha}
Suppose $s =_\alpha s$ and $\gamma,\gamma'$ are substitutions on the same domain such that $\gamma(x) =_\alpha \gamma'(x)$ for all $x$ in $\domain(\gamma)$.
Suppose that $s\gamma = t$ and $s'\gamma' = t'$.
Then $t =_\alpha t'$.
\end{lemma}

(Hence, if $\gamma = \gamma'$ we have $s\gamma =_\alpha s'\gamma$.)

\begin{proof}
By Lemma \ref{lem:substitutionalphahelper}(\ref{lem:substitutionalphahelper:subst}), choosing
$p = k = 0$, $\mu = \mu_0$, $\xi = \xi_0$, $\nu = \nu_0$, $\chi = \chi_0$.
Since $\mu_0(x) = 0$ for all $x$, the only requirement on the substitution is that $\gamma(x)
=_\alpha^{\nu_0,\chi_0,1} \gamma'(x)$ for all $x \in \FV(s)$, which holds by assumption.
\end{proof}

Next, we demonstrate that combining substitutions works as we may expect.

\begin{lemma}\label{lem:combinesubst}
Always $s(\gamma\delta) =_\alpha (s\gamma)\delta$.
\end{lemma}

\begin{proof}
Let $\eta = \gamma\delta$; that is, $\domain(\eta) = \domain(\gamma) \cup \domain(\delta)$ and
$\eta(x) = \gamma(x)\delta$ for all $x \in \domain(\eta)$ and $\eta(x) = \delta(x)$ for aother $x$
in its domain.  We will prove that there exists $t$ such that $s\gamma = t$ and $t\delta = s\eta$.
The desired result then follows by Lemma \ref{lem:substitutionalpha}.
\end{proof}

We can now prove several results regarding $\alpha$-equivalence.

To start, we will work with a very practical result that allows us to limit interest to
\emph{well-behaved} terms $s$, which have the following properties:

\begin{itemize}
\item For every position $p$ in $s$: $\FV(s|_p) \cap \BV(s|_p) = \emptyset$.
\item For every position $p$ in $s$: if $s|_p = \abs{\avar}{s'}$ then $\avar \notin \BV(s')$.
\end{itemize}

That is, the same variable cannot occur both bound and free in a term, and a variable $\avar$
cannot be bound again in the subterm $s$ of $\abs{\avar}{s}$.  If a term violates this restriction,
it is always possible to find an $=_\alpha$-equal term that does satisfy it, as we see below:

\begin{lemma}\label{lem:wellbehaved}
For all terms $s$ there exists a well-behaved term $s'$ such that $s =_\alpha s'$.
\end{lemma}

\begin{proof}
Let $\mathit{Safe} = \Vbound \setminus (\FV(s) \cup \BV(s))$.  Since $\FV(s)$ and $\BV(s)$ are
necessarily finite, $\mathit{Safe}$ contains infinitely many variables of all types.  We will
construct a well-behaved term whose bound variables are all taken from $\mathit{Safe}$.

Let $\avar_1,\dots,\avar_k \in \Vbound$ (not necessarily all distinct), and $\cvar_1,\dots,\cvar_k$
be distinct variables in $\mathit{Safe}$ such that for all $i$: $\avar_i$ and $\cvar_i$ have the
same type.  Let $\mu_k,\xi_k : \Vbound \mapsto \{1,\dots,k\}$ be such that:
\begin{itemize}
\item for $\bvar \notin \{\avar_1,\dots,\avar_k\}$: $\mu_k(\bvar) = 0$;
\item for $\bvar \notin \{\cvar_1,\dots,\cvar_k\}$: $\xi_k(\bvar) = 0$;
\item for $1 \leq i \leq k$: if $\mu_k(\avar_i) = j$ then $\avar_i = \avar_j$;
\item for $1 \leq i \leq k$: $\xi_k(\avar_i) = i$.
\end{itemize}
Let $t \subtermeq s$.  By induction on the form of $t$, we will construct a term $t'$ such that
(1) $t'$ is well-behaved; (2) $\BV(t') \subseteq \mathit{Safe} \setminus \{\cvar_1,\dots,\cvar_k
\}$, (3) $\FV(t') \cap \mathit{Safe} \subseteq \{\cvar_1,\dots,\cvar_k\}$, (4) $t'$ has the same
type as $t$, and (5) $t =_\alpha^{\mu_k,\xi_k,k+1} t'$ and $\BV(t') \cap \{\cvar_1,\dots,\cvar_k\}
= \emptyset$.  This proves the lemma since it gives a well-behaved $s'$ such that
$s =_\alpha^{\mu_0,\xi_0,1} s'$ (for the case $t = s$ and $k = 0$).

Consider the form of $t$.

If $t = \afun(t_1,\dots,t_n)$ or $t = \bvar(t_1,\dots,t_n)$, then each $t_i \subtermeq s$ (since
$t = s|_p$ implies $t_i = s|_{p \cdot i}$).  By the induction hypothesis, we obtain $t_1',\dots,
t_n'$ which satisfy (1--3) and also that $t_i'$ has the same type as $t_i$ and $t_i
=_\alpha^{\mu_k,\xi_k,k+1} t_i'$ for $1 \leq i \leq n$.
We let:
\begin{itemize}
\item $t' := \afun(t_1',\dots,t_n')$ if $t = \afun(t_1,\dots,t_n)$;
\item $t' := \bvar(t_1',\dots,t_n')$ if $t = \bvar(t_1,\dots,t_n)$ with $\mu_k(\bvar) = 0$;
\item $t' := \cvar_i(t_1',\dots,t_n')$ if $t = \bvar(t_1,\dots,t_n)$ and $\mu_k(\bvar) = i > 0$.
\end{itemize}
We will see that this satisfies requirements (1--5).  First, item (4): since each $t_i'$ is
well-typed and has the same type as $t_i$, clearly $t'$ is well-typed and has the same type as
$t$ in the first two cases; in the final case, this holds because $\cvar_i$ has the same type as
$\avar_i$ and if $\mu_k(\bvar) = i$ then $\bvar = \avar_i$.  Item (5) holds obviously by definition
of $=_\alpha^{\mu_k,\xi_k,k+1}$.  Item (2) follows because $\BV(t') = \BV(t_1') \cup \dots \cup
\BV(t_n')$.  Item (3) follows because $\FV(t') = \FV(t_1') \cup \dots \cup \FV(t_n') \cup A$ where
either (a) $A = \emptyset$, (b) $A = \{\bvar\}$ or (c) $A = \{\cvar_i\}$.  We know by the induction
hypothesis that the elements of $\FV(t_1') \cup \dots \FV(t_n')$ are in $\mathit{Safe} \setminus
\{\cvar_1,\dots,\cvar_k\}$, so in case (a) we are done, in case (b) we note that $\bvar$ occurs in
$s$ (be it free or bound) so $\bvar \notin \mathit{Safe}$, and in case (c) we have $\cvar_i \in
\{\cvar_1,\dots,\cvar_k\}$.
As for requirement (1): well-behavedness is satisfied for all $p \neq \epsilon$, since all
immediate subterms are well-behaved; hence, it suffices to see that $\FV(t) \cap \BV(t) =
\emptyset$.  But this follows immediately from (2) and (3).

Alternatively, $t = \abs{\bvar}{u}$, in which case $\bvar \in \BV(s)$.  Let $\cvar_{k+1}$ be a
fresh variable in $\mathit{Safe}$ of the same type.  Let $\mu_{k+1} := \mu_k[\bvar := k+1]$ and
$\xi_{k+1} := \mu_k[\cvar_{k+1}:=k+1]$.  Note that $u \subseteq s$ as well.  Hence, we may
apply the induction hypothesis to obtain $u'$ which satisfies (1--3), has the same type as $u$
and has $u =_\alpha^{\mu_{k+1},\xi_{k+1},k+2} u'$.  Let $t' := \abs{\cvar_{k+1}}{u'}$.  Then
clearly $t'$ is well-typed and has the same type as $t$, and $t =_\alpha^{\mu_k,\xi_k,k+1} t'$
holds by definition.  $\BV(t') = \BV(u') \cup \{\cvar_{k+1}\}$, and since $\cvar_{k+1} \in
\mathit{Safe} \setminus \{\cvar_1,\dots,\cvar_k\}$ (2) holds by the induction hypothesis.
$\FV(t') \cap \mathit{Safe} = (\FV(u') \setminus \{ \cvar_{k+1} \}) \cap \mathit{Safe} =
(\FV(u') \cap \mathit{Safe}) \setminus \{\cvar_{k+1}\}$, which by the induction hypothesis
$\subseteq \{\cvar_1,\dots,\cvar_{k+1}\} \setminus \{\cvar_{k+1}\} = \{\cvar_1,\dots,\cvar_k\}$.
Only well-behavedness of $t'$ remains to be shown, and since the immediate subterm $u'$ is
well-behaved by the induction hypothesis, we are done if (a) $\FV(t') \cap \BV(t') = \emptyset$
and (b) $\cvar_{k+1} \notin \BV(u')$.  But (2) implies (b) and (2) and (3) together imply (a).
\end{proof}



\subsubsection{A practical result}

We have the following practical result that allows us to verify one abstraction is an instance of
another:

\begin{lemma}
If $\abs{x}{s}$ and $\abs{y}{t}$ satisfy the restriction given in Section\ref{subsec:termvars}, we
have:
$(\abs{x}{s})\gamma =_\alpha \abs{y}{t}$ if and only if
$s ([x:=y] \cup [z:=\gamma(z) \mid z \in \domain(\gamma) \wedge z \neq x]) =_\alpha t$.
\end{lemma}

\begin{proof}
For a given variable $u$, let $\delta_u := [x:=u] \cup [z:=\gamma(z) \mid z \in \domain(\gamma)
\wedge z \neq x]$.  The lemma states that $(\abs{x}{s})\gamma =_\alpha \abs{y}{t}$ if and only if
$s\delta_y =_\alpha t$.

First suppose that $s\delta_y =_\alpha t$.

=====================

By definition, $(\abs{x}{s})\gamma = \abs{u}{(s\delta_u)}$ for a fresh variable $u$.  If $y \neq x$
and $y$ does not occur in $s$ or the domain or range of $\gamma$, then $y$ is fresh, so we can just
choose $y$ for $u$ and obtain
\end{proof}

\subsection{Rules and rewriting}

A rule $\rho$ is a pair $\ell \arrz r$ of two terms with the same type.

A rule $\rho = \ell \arrz r$ can generate a set of term-pairs in two ways:
\begin{itemize}
\item \emph{default matching}: $\mathit{Pairs}_\rho := \{ (u,v) \mid u,v \in \Terms(\F,\V) \wedge$
  there exist a substitution $\gamma : \Vfree \to \Terms(\F,\V)$ and $w_1,\dots,w_n \in
  \Terms(\F,\V)$ such that $u = (\ell\gamma) \cdot [w_1,\dots,w_n]$ and
  $v = (r\gamma) \cdot [w_1,\dots,w_n] \}$
\item \emph{plain matching}: $\mathit{Pairs}_\rho^{\mathtt{plain}} := \{ (u,v) \mid u,v \in
  \Terms(\F,\V) \wedge$ there exist a substitution $\gamma : \Vfree \to \Terms(\F,\V)$ and $w_1,
  \dots,w_n \in \Terms(\F,\V)$ such that $u = (\ell\gamma) \cdot [w_1,\dots,w_n]$ and
  $v = (r\gamma) \cdot [w_1,\dots,w_n]$ and if $u \suptermeq x(u_1,\dots,u_k)$ with $x \in
  \Vfree$ and $n > 0$ then $\gamma(x)$ is not an abstraction, and if $n > 0$ then neither
  $u$ nor $v$ is an abstraction$\}$.
\end{itemize}
So default matching allows any instance of the left-hand side of a rule to be replaced by the
corresponding instance of the right-hand side. Matching is -- essentially -- modulo
$\beta$-reduction, which means that a term $X(\mathtt{a})$ with $X \in \Vfree$ and
$\mathtt{a} \in \F$ can be instantiated by any term $s$ of the same type (since we could
choose $\gamma(X) = \abs{y}{s}$ with $y$ a fresh variable).
On the other hand, plain matching does not allow this: neither substitution nor the
application $[w_1,\dots,w_n]$ is allowed to create a situation where an abstraction is applied
to another term. Thus, $X(\mathtt{a})$ can only be instantiated to terms of the form
$x(s_1,\dots,s_k,\mathtt{a})$ or $\afun(s_1,\dots,s_k,\mathtt{a})$.

For a given set of rules $\Rules$ and $T \subseteq \Terms(\F,\V)$, the reduction relation
$\arr{\Rules,T}$ is given by:
\begin{itemize}
\item for $s,t \in T$ and $\rho \in \Rules$: if there exist a pair $(u,v) \in
  \mathit{Pairs}_{\rho}$ and a position $p$ such that
  $s|_p = u$ and $s[v]_p = t$, then $s \arr{\Rules,T} t$
\end{itemize}
The \emph{plain} reduction relation is given by:
\begin{itemize}
\item for $s,t \in T$ and $\rho \in \Rules$: if there exist a pair $(u,v) \in
  \mathit{Pairs}_{\rho}^{\mathtt{plain}}$ and a position $p$ such that
  $s|_p = u$ and $s[v]_p = t$, then $s \arr{\Rules,T}^{\mathtt{plain}} t$
\end{itemize}

\medskip
We will denote $\arr{\Rules}$ for the relation $\arr{\Rules,\Terms(\F,\V)}$.

A rule is a \emph{pattern rule} if the left-hand side $\ell$ is a pattern. Pattern rules are
particularly important for default matching, and do not play a significant role in plain matching.

\subsection{HOTRSs}

We now have all the ingredients to define a \emph{higher-order term rewriting system (HOTRS)}.

\mysubsection{Abstract Rewriting Systems}

An abstract rewriting system is a pair $(\mathcal{A},\arrz)$ where $\mathcal{A}$ is a set and
$\arrz$ a binary relation on that set.  Properties such as termination and confluence can be
expressed in terms of abstract rewriting systems.

\mysubsection{HOTRSs}

A higher-order term rewriting system (HOTRS) is a triple $(T,\Rules_d,\Rules_p)$ representing
an abstract rewriting system of the form $(T,\arr{\Rules_d,T} \cup
\arr{\Rules_p}^{\mathtt{plain}},T)$; here, we require that $T \subseteq \Terms(\F,\V)$ and for
all rules $\ell \arrz r \in \Rules_p$: both $\ell$ and $r$ are in $T$.

\mysubsection{Examples of HOTRSs}
Many standard forms of term rewriting systems can be expressed as HOTRSs.

A \emph{many-sorted term rewriting system} (MTRS) is a HOTRS $(\FOTerms(\F,\V), \emptyset,
\Rules)$ (so with no rules in $\Rules_d$), which satisfies the following properties:
\begin{itemize}
\item for all $(\afun : \atype) \in \F$: $\order(\atype) \leq 1$;
\item for all $\ell \arrz r \in \Rules$: $\ell$ is not a variable, and $\FV(r) \subseteq \FV(\ell)$.
\end{itemize}
It is worth noting that, in a many-sorted term rewriting system, $\arr{\Rules}$ and
$\arr{\Rules}^{\mathtt{plain}}$ define the same relation because the left-hand sides of rules do
not have variables that occur at the head of an application, and both left- and all rules have
a base type. Therefore, an MTRS $(\FOTerms(\F,\V),\emptyset,\Rules)$ is essentially the same as
$(\FOTerms(\F,\V),\Rules,\emptyset)$.
Moreover, we have the following property: if $s \in \FOTerms(\F,\V)$ and $s \arr{\Rules} t$
then also $t \in \FOTerms(\F,\V)$. Thus, when analysing reductions, only the set of start terms
needs to be constrained.

An \emph{unsorted first-order term rewriting system} (TRS) is a many-sorted term rewriting system
with $\Sorts = \{ \unitsort \}$.

An \emph{applicative term rewriting system} (ATRS) is a HOTRS $(\ATerms(\F,\V),\emptyset,\Rules)$.
Note that such an ATRS \emph{is} actually different from the similar HOTRS
$(\ATerms(\F,\V),\Rules,\emptyset)$: in a system $(T,\Rules,\emptyset)$, a rule
$\afun(\avar(\nul)) \arrz \afun(\avar(\symb{1}))$ will reduce $\afun(\symb{2})$ to itself
(through the substitution $\gamma = [\avar:=\abs{\bvar}{\symb{2}}]$), while in the ATRS
$(T,\emptyset,\Rules)$ this cannot happen.  However, if all rules in $\Rules$ are
\emph{pattern rules}, then the two systems are the same once more.

A \emph{higher-order rewriting system} (HRS) is a HOTRS of the form
$\{ s \in \Terms(\F,\V) \mid s$ is in $\eta$-long form$\},\ \Rules,\ \emptyset)$ where for all
rules $\ell \arrz r \in \Rules$ also $\ell$ and $r$ are in $\eta$-long form.

A \emph{pattern higher-order rewriting system} (PRS) is a HRS where moreover all elements of
$\Rules$ are pattern rules.\footnote{TODO: extend the definition of a pattern, since this
currently disallows third-order PRSs.}

An \emph{algebraic functional system} (AFS) is a HOTRS $(T,\Rules_\beta,\Rules)$ with the
following properties:
\begin{itemize}
\item $\F \supseteq \{ @_{\sigma,\tau} : (\sigma \arrtype \tau) \arrtype \sigma \arrtype \tau
  \mid \sigma,\tau \in \Types \}$;
\item there is a function $\mathit{arity} : \F \mapsto \mathbb{N}$ such that
  $\mathtt{arity}(@_{\sigma,\tau}) = 2$ for all $\sigma,\tau$ and:
\item[] $T = \{ s \in \Terms(\F,\V) \mid \forall t \subtermeq s: t$ has the form
  $\abs{x}{t'}$ or $x$ or $\afun(s_1,\dots,s_{\arity(\afun)})\}$;
\item $\Rules_\beta := \{ @_{\sigma,\tau}(\abs{\avar}{\bvar(\avar)},\cvar) \arrz \bvar(\cvar)
  \mid \sigma,\tau \in \Types \}$; here, $\avar \in \Vfree$ and $\bvar,\cvar \in \Vbound$.
\end{itemize}
Note that the left- and right-hand sides of the rules in $\Rules_\beta$ are not in $T$ (since
variable applications are not allowed in afs-terms), but this is explicitly allowed by the
definition of a HOTRS. Note also that in an AFS, all rules are necessarily pattern rules, and
consequently $\arr{\Rules}^{\mathtt{plain}}$ and $\arr{\Rules}$ define the same relation.
Thus, here the underlying abstract rewriting system is the same as the HOTRS $(T,\Rules \cup
\Rules_\beta,\emptyset)$.

\mysubsection{Discussion} The reader may have noted that almost all of the different cases
above can be represented as a HOTRS of the form $(T,\Rules,\emptyset)$; the only case where
plain rules are \emph{necessary} is for ATRSs with non-pattern rules.

The reason that we do consider this broader definition where both $\Rules_d$ and $\Rules_p$ may
be supplied is to allow for a broad range of HOTRS transformations, which are useful for
analysis purposes.

\appendix

\section{Correctness of the unconstrained formalism}

\subsection{$\alpha$-equality}

We first see that $=_\alpha$ is an equivalence relation, so that we can reason modulo it!

\begin{lemma}\label{lem:alphaequiv}
For all terms $s,t,q$, functions $\mu,\xi,\chi : \V \to \N$ and $k \in \N$:
\begin{enumerate}
\item\label{lem:alphaequiv:reflexive}
  $s =_\alpha^{\mu,\mu,k} s$;
\item\label{lem:alphaequiv:symmetric}
  if $s =_\alpha^{\mu,\xi,k} t$ then $t =_\alpha^{\xi,\mu,k} s$;
\item\label{lem:alphaequiv:transitive}
  if $s =_\alpha^{\mu,\xi,k} t$ and $t =_\alpha^{\xi,\chi,k} q$ then $s =_\alpha^{\mu,\chi,k} q$.
\end{enumerate}
\end{lemma}

\begin{proof}
All follow by a straightforward induction on the size of $s$. [Outcommented]
%
%(\ref{lem:alphaequiv:reflexive})
%If $s = \afun(s_1,\dots,s_n)$ or $s = x(s_1,\dots,s_n)$ then by IH all $s_i =_\alpha^{\mu,\mu,k}
%s_i$.
%In both cases, and whether $\mu(x) = 0$ or $\mu(x) = \mu(x) > 0$, $s =_\alpha^{\mu,\mu,k} s$
%follows directly.
%If $s = \abs{x}{s'}$ then by IH $s' =_\alpha^{\mu[x:=k],\mu[x:=k],k+1} s'$; here also
%$s =_\alpha^{\mu,\mu,k} s$ follows directly.
%
%(\ref{lem:alphaequiv:reflexive})
%If $s = \afun(s_1,\dots,s_n)$ then necessarily $t = \afun(t_1,\dots,t_n)$ with $s_i =_\alpha^{\mu,
%\xi,k} t_i$ for all $i$; by IH also $t_i =_\alpha^{\xi,\mu,k} s_i$, so the lemma follows.
%If $s = x(s_1,\dots,s_n)$ then $t = y(t_1,\dots,t_n)$ with $s_i =_\alpha^{\mu,\xi,k} t_i$ (and
%therefore, by IH, also $t_i =_\alpha^{\xi,\mu,k} s_i$) and either $\mu(x) = \xi(x) = 0$ and $x = y$,
%or $\mu(x) = \xi(y) > 0$. Either way, $t =_\alpha^{\xi,\mu,k}$ follows immediately.
%If $s = \abs{x}{s'}$ then $t = \abs{y}{t'}$ and $s' =_\alpha^{\mu[x:=k],\xi[y:=k],k+1} t'$. By IH
%also $t' =_\alpha^{\xi[y:=k],[x:=k],k+1} s'$.
%
%(\ref{lem:alphaequiv:transitive})
%If $s = \afun(s_1,\dots,s_n)$ then $t = \afun(t_1,\dots,t_n)$ and $q = \afun(q_1,\dots,q_n)$ with
%$s_i =_\alpha^{\mu,\xi,k} t_i$ and $t_i =_\alpha^{\xi,\chi,k} q_i$ for all $i$; we complete by the
%induction hypothesis.
%If $s = x(s_1,\dots,s_n)$ and $\mu(x) = 0$ then $t = x(t_1,\dots,t_n)$ with $\xi(x) = 0$, so $q =
%x(q_1,\dots,q_n)$ with $\chi(x) = 0$, and $s_i =_\alpha^{\mu,\xi,k} t_i =_\alpha^{\xi,\chi,k} q_i$;
%we again complete by the induction hypothesis.
%If $s = x(s_1,\dots,s_n)$ and $\mu(x) > 0$ then $t = y(t_1,\dots,t_n)$ with $\xi(y) = \mu(x) > 0$,
%so $q = z(q_1,\dots,q_n)$ with $\chi(z) = \xi(y) > 0$. We again complete by the IH.
\end{proof}

Choosing $k = 1$ and $\mu,\xi,\chi$ the function mapping everything to $0$, we obtain:

\begin{corollary}\label{corr:alphaequiv}
$=_\alpha$ is an equivalence relation.
\end{corollary}

As noted in the text, $\FV$ defines a function on equivalence classes:

\begin{lemma}\label{lem:alphafreevar}
If $s =_\alpha^{\mu,\xi,k} t$ then $\FV(s) \setminus \{ x \mid \mu(x) \neq 0 \} = \FV(t) \setminus \{ x \mid \xi(x) \neq 0 \}$.
\end{lemma}

\begin{proof}
By induction on the size of $s$.
All cases are straightforward. [Outcommented]
    % For brevity, we denote $M := \{ x \mid \mu(x) \neq 0 \}$ and $X :=  \{ x \mid \xi(x) \neq 0 \}$.
    % \begin{itemize}
    % \item If $s = \afun(s_1,\dots,s_n)$ then $t = \afun(t_1,\dots,t_n)$ and each $s_i =_\alpha^{\mu,\xi,k} t_i$.
    %   By the induction hypothesis, $\FV(s_i) \setminus M = \FV(t_i) \setminus X$.
    %   Since $\FV(s) \setminus M = (\FV(s_1) \setminus M) \cup \dots \cup (\FV(s_n) \setminus M)$ for any set $M$, and similar for $\FV(t) \setminus X$, we are done.
    % \item If $s = x(s_1,\dots,s_n)$ and $\mu(x) = 0$ then $t = x(t_1,\dots,t_n)$ and $\xi(x) = 0$ as well.
    %   Therefore $x \notin M$ and $x \notin X$.
    %   Hence $\FV(s) \setminus M = \{ x \} \cup (\FV(s_1) \setminus M) \cup \dots \cup (\FV(s_n) \setminus M)$ and
    %   $\FV(t) \setminus X = \{ x \} \cup (\FV(t_1) \setminus X) \cup \dots \cup (\FV(t_n) \setminus X)$.
    %   We are done by the induction hypothesis.
    % \item If $s = x(s_1,\dots,s_n)$ and $\mu(x) > 0$ then $t = y(t_1,\dots,t_n)$ and $\xi(x) = mu(x) > 0$ as well.
    %   Therefore $x \in M$ and $x \in X$.
    %   Hence $\FV(s) \setminus M = (\FV(s_1) \setminus M) \cup \dots \cup (\FV(s_n) \setminus M)$ and
    %   $\FV(t) \setminus X = \cup (\FV(t_1) \setminus X) \cup \dots \cup (\FV(t_n) \setminus X)$.
    %   We complete once more with the induction hypothesis.
    % \item If $s = \abs{x}{s'}$ then $t = \abs{y}{t'}$ and $s' =_\alpha^{\mu[x:=k],\xi[y:=k],k+1} t'$.
    %   We have $\FV(s) \setminus M = (\FV(s') \setminus \{ x \}) \setminus M = \FV(s') \setminus \{ z \mid \mu[x:=k](z) \neq 0 \}$.
    %   Moreover, $\FV(t) \setminus X = (\FV(t') \setminus \{ z \mid \xi[y:=k](z) \neq 0 \}$.
    %   We complete again by the induction hypothesis.
    %   \qedhere
    % \end{itemize}
\end{proof}

Choosing $k = 1$ and $\mu,\xi$ the empty functions again, we obtain:

\begin{corollary}\label{corr:alphafreevar}
If $s =_\alpha t$ then $\FV(s) = \FV(t)$.
\end{corollary}

Next, we define some helper results that have little meaning on their own, but will prove useful
when reasoning about $\alpha$-equivalence (especially in combination with substitution).

\begin{lemma}\label{lem:alphaincrease}
If $s =_\alpha^{\mu,\chi,k} t$ then $s =_\alpha^{\mu,\chi,k+1} t$.
\end{lemma}

\begin{proof}
For an integer $0 < n \leq k$, let $\mathit{up}_n(i) = i$ for $i \leq n$ and $\mathit{up}_n(i) = i+1$ for $i > n$.
Let $\mu_n(x) = \mathit{up}(\mu(x))$ and $\xi_n(x) = \mathit{up}_n(x)$.
We will prove by induction on $s$ that for fixed $n > 0$, all $k \geq n$: if $s =_\alpha^{\mu,\chi,k} t$ then $s =_\alpha^{\mu_n,\chi_n,k+1} t$.
The proof is straightforward. [Outcommented]
    % \begin{itemize}
    % \item If $s = \afun(s_1,\dots,s_m)$ then $t = \afun(t_1,\dots,t_m)$ with each $s_i =_\alpha^{\mu,\chi,k} t_i$ so $s_i =_\alpha^{\mu_n,\chi_n,k+1} t_i$ by the IH;
    %   hence $s =_\alpha^{\mu_n,\chi_n,k+1} t$.
    % \item If $s = x(s_1,\dots,s_m)$ and $\mu(x) = 0$ then $t = x(t_1,\dots,t_m)$ and $\xi(x) = 0$ and we complete as above.
    % \item If $s = x(s_1,\dots,s_m)$ and $\mu(x) > 0$, then $t = y(t_1,\dots,t_m)$ and $\xi(y) = \mu(x)$.  But then $\mathit{up}(\mu(x)) = \mathit{up}(\xi(y))$ as well,
    %   so $\mu_n(x) = \xi_n(y)$.  Also $s_i =_\alpha^{\mu_n,\chi_n,k+1} t_i$ for each $i$ by the IH, so indeed $s =_\alpha^{\mu_n,\chi_n,k+1} t$.
    % \item If $s = \abs{x}{s'}$ then $t = \abs{y}{t'}$ and $s' =_\alpha^{\mu[x:=k],\xi[y:=k],k+1} t'$. Since $k \geq n$, certainly
    %   $(\mu[x:=k])_n = \mu_n[x:=k+1]$ and $(\xi[y:=k])_n = \xi_n[y:=k+1]$.  Hence, by the induction hypothesis,
    %   $s' =_\alpha^{\mu_n[x:=k+1],\xi_n[x:=k+1],k+2} t'$ and therefore $s =_\alpha^{\mu_n,\xi_n,k+1} t$.
    %   \qedhere
    % \end{itemize}
\end{proof}

\begin{lemma}\label{lem:alphaunusedvar}
Suppose $s =_\alpha^{\mu,\xi,k} t$ and $\mu'(x) = \mu(x)$ for all $x \in \FV(s)$, and $\xi'(y) = \xi(y)$ for all $y \in \FV(t)$.
Then $s =_\alpha^{\mu',\xi',k} t$.
\end{lemma}

\begin{proof}
By induction on the size of $s$.
All cases are straightforward. [Outcommented]
    % \begin{itemize}
    % \item If $s = \afun(s_1,\dots,s_n)$ then $t = \afun(t_1,\dots,t_n)$ and each $s_i =_\alpha^{\mu,\xi,k} t_i$.
    %   By the induction hypothesis, each $s_i =_\alpha^{\mu',\xi',k} t_i$.
    %   This immediately gives $s =_\alpha^{\mu',\xi',k} t$ as required.
    % \item If $s = x(s_1,\dots,s_n)$ and $\mu(x) = 0$, then $t = x(t_1,\dots,t_n)$ and $\xi(x) = 0$.
    %   Since clearly $x \in \FV(s) \cap \FV(t)$ we have $\mu'(x) = \xi'(y) = 0$ as well.
    %   Hence $s =_\alpha^{\mu',\xi',k} t$ by the induction hypothesis on each $s_i$.
    % \item If $s = x(s_1,\dots,s_n)$ and $\mu(x) > 0$ then $t = y(t_1,\dots,t_n)$ with $\xi(y) = \mu(x)$.
    %   Since clearly $x \in \FV(s)$ and $y \in \FV(t)$ we have $\mu'(x) = \mu(x) = \xi(x) = \xi'(x) > 0$ as well.
    %   We again completely easily by the induction hypothesis.
    % \item If $s = \abs{x}{s'}$ then $t = \abs{y}{t'}$ and $s' =_\alpha^{\mu[x:=k],\xi[y:=k],k+1} t'$.
    %   Now,for all $z \in \FV(s')$: either $z = x$ and $\mu'[x:=k](z) = k = \mu[x:=k](z)$, or $z \in \FV(s)$ and $\mu[x:=k](z) = \mu(z) = \mu'(z) = \mu'[x:=k](z)$ by assumption.
    %   Similarly, for all $z \in \FV(t')$ we have $\xi'[x:=k](z) = \xi[x:=k](z)$.
    %   Hence we can apply the induction hypothesis on $s'$ to obtain $s' =_\alpha^{\mu'[x:=k],\xi'[y:=k],k+1} t'$.
    %   This immediately implies $s =_\alpha{\mu',\xi',k} t$.
    %   \qedhere
    % \end{itemize}
\end{proof}

\subsection{Well-definedness of substitution and application}

Avoiding the suggestive equality notation, let us reformulate the notions of application and substitutions as \emph{relations} as follows:

\begin{enumerate}
\item $\aprel{s, [], s}$ if $m = 0$;
\item\label{ap:combi} $\aprel{s, [t_1,\dots,t_{m+1}], q}$ if there is $u$ such that $\aprel{s,t_1,u}$ and $\aprel{u, [t_2,\dots,t_{m_1}], q}$;
\item $\aprel{\afun(s_1,\dots,s_n),t,\afun(s_1,\dots,s_n,t)}$;
\item $\aprel{\avar(s_1,\dots,s_n),t,\avar(s_1,\dots,s_n,t)}$;
\item\label{ap:abs} $\aprel{\abs{\avar}{s},t,q}$ if $\subrel{s,[\avar:=t],q}$;
\item\label{subst:func} $\subrel{\afun(s_1,\dots,s_n),\gamma,\afun(t_1,\dots,t_n)}$ if $\subrel{s_i,\gamma,t_i}$ for $1 \leq i \leq n$;
\item\label{subst:safevar} $\subrel{\avar(s_1,\dots,s_n),\gamma,\avar(t_1,\dots,t_n)}$ if $x \notin \domain(\gamma)$ and $\subrel{s_i,\gamma,t_i}$ for $1 \leq i \leq n$;
\item\label{subst:appvar} $\subrel{\avar(s_1,\dots,s_n),\gamma,q}$ if there exist $t_0,\dots,t_n$ such that $\gamma(x) = t_0$ and $\subrel{s_i,\gamma,t_i}$ for
  $1 \leq i \leq n$ and $\aprel{t_0,[t_1,\dots,t_n],q}$
\item\label{subst:abs} $\subrel{\abs{\avar}{s},\gamma,\abs{\cvar}{q}}$ if $\cvar$ has the same type as $\avar$ and:
  \begin{itemize}
  \item $\cvar \notin \FV(\abs{\avar}{s}) \setminus \domain(\gamma)$;
  \item $\cvar \notin \FV(\gamma(y))$ for any $y \in \domain(\gamma) \cap \FV(\abs{\avar}{s})$;
  \item $\subrel{s,[\avar:=\cvar] \cup [\bvar := \gamma(\bvar) \mid \bvar \in \domain(\gamma) \setminus \{\avar\}],q}$.
  \end{itemize}
\end{enumerate}
It is easy to see that this defines the same relation as application and substitution in the main text.

We can now specify and prove the well-definedness result:

\begin{lemma}\label{lem:substdefined}
The following results hold:
\begin{itemize}
\item For every term $s$ and substitution $\gamma$ there exists $q$ with $\subrel{s,\gamma,q}$. %$s\gamma = q$.
\item For every term $s : \atype_1 \arrtype \dots \atype_m \arrtype \btype$ and terms
  $t_1 : \atype_1,\dots,t_m : \atype_m$: there exists $q$ with $\aprel{s, [t_1,\dots,t_m], q}$. %$s \cdot [t_1,\dots,t_m] = q$.
\end{itemize}
\end{lemma}

\begin{proof}
Let $K_{s,\gamma} = 1 + 2 * \max\{ \order(\btype) \mid (\bvar : \btype) \in \FV(s) \wedge \bvar \in
\domain(\gamma) \wedge \gamma(\bvar) \notin \V \}$, and $L_s = 2 * \order(\atype)$ if $s : \atype$.
We prove the results by mutual induction on $K_{s,\gamma}$ or $L_s$ first, $m$ second (for the
second statement) and the size of $s$ third.

For the first statement:
\begin{itemize}
\item If $s = \afun(s_1,\dots,s_n)$ or $x(s_1,\dots,s_n)$, then by the third induction hypothesis,
  there are $t_1,\dots,t_n$ so that $\subrel{s_i,\gamma,t_i}$ for $1 \leq i \leq n$.
  \begin{itemize}
  \item If $s = \afun(s_1,\dots,s_n)$ we hence complete by case \ref{subst:func}.
  \item If $s = x(s_1,\dots,s_n)$ with $x \notin \FV(\gamma)$ by case \ref{subst:safevar}.
  \item If $s = x(s_1,\dots,s_n)$ and $y := \gamma(x) \in \V$, then we complete by case
    \ref{subst:appvar} by choosing $q := y(t_1,\dots,t_n)$
  \item Otherwise, let $\avar : \atype$ and $k := \order(\atype)$.  By definition, $K_{s,\gamma}
    \geq 1 + 2 * k > 2 * k = L_{\gamma(\avar)}$.  We find $q$ with
    $\aprel{\gamma(\avar),[t_1,\dots,t_n],q}$ by the first induction hypothesis,
    and complete with case \ref{subst:appvar}.
  \end{itemize}
\item If $s = \abs{\avar}{s'}$ then first note that a suitable $\cvar$ can always be found, as
  $\Vbound$ has infinitely many variables of all types and $\FV(s) \cap \domain(\gamma)$ is
  finite. Then, writing $\delta := [\avar:=\cvar] \cup [\bvar:=\gamma(\bvar) \mid \bvar
  \in \domain(\gamma) \setminus \{\avar\}]$, we have $K_{s,\gamma} \geq K_{s',\delta}$: the only
  variable in $\FV(s') \setminus \FV(s)$ is $\avar$, which has $\delta(\avar) \in \V$.  Hence,
  here we can also apply the third induction hypothesis and complete by case \ref{subst:abs}.
\end{itemize}
For the second statement:
\begin{itemize}
\item If $s = \afun(s_1,\dots,s_n)$ then $\aprel{s, [t_1,\dots,t_m], \afun(s_1,\dots,s_n,t_1,\dots,
  t_m)}$.
\item Similar if $s = \avar(s_1,\dots,s_n)$.
\item If $s = \abs{\avar}{s'}$, then $s : \atype \arrtype \btype$. We are done if $m = 0$, and if
  $m > 1$ we note that $\order(\atype \arrtype \btype) \geq \order{\atype},\order(\btype)$, so by
  the second induction hypothesis there exist $u$ and $q$ such that $\aprel{s, [t_1], u}$ and
  $\aprel{u, [t_2,\dots,t_m], q}$, giving $\aprel{s, [t_1,\dots,t_m], q}$ by case \ref{ap:combi}.
  The case remains where $m = 1$, so $s \cdot [t_1] = s'[\avar:=t]$.
  Then $\avar$ has type $\atype$, and $\order(\atype) < \order(\atype \arrtype \btype)$.
  Since $\avar$ is the only variable in the domain of $[\avar:=t]$, we find $q$ such that
    $\subrel{s',[x:=t],q}$ with the first induction hypothesis.
  \qedhere
\end{itemize}
\end{proof}

Towards the second result for well-definedness, that substitution and application define functions
on equivalence classes, we specify the following lemma.  This is formulated more generally than we
need to obtain an easier induction.

\begin{lemma}\label{lem:substitutionalpha}
Let $k$ be an integer, and $\nu,\chi : \V \mapsto \{0,\dots,k\}$.  We have:
\begin{enumerate}
\item\label{lem:substitutionalpha:subst}
  Assume given terms $s,s'$, an integer $p$, mappings $\mu,\xi : \V \mapsto \{0,\dots,p\}$ and substitutions $\gamma,\gamma'$ such that:
  \begin{itemize}
  \item for all $x \in \FV(s)$ with $\mu(x) = 0$ we have:
    $\gamma(x) =_\alpha^{\nu,\chi,k+1} \gamma'(x)$ \\
    (where we let $\gamma(x) = x$ if $x \notin \domain(\gamma)$);
  \item for all $x \in \FV(s)$, $y \in \FV(s')$ such that $\mu(x) = \xi(y) > 0$ we have:
    $\gamma(x) =_\alpha^{\nu,\chi,k+1} \gamma'(y)$
  \item $s =_\alpha^{\mu,\xi,p+1} s'$
  \end{itemize}
  \ \\
  Assume given $t,t'$ such that $\subrel{s,\gamma,t}$ and $\subrel{s',\gamma',t'}$.  Then $t =_\alpha^{\nu,\chi,k+1} t'$.
\item\label{lem:substitutionalpha:appl}
  Assume given terms $t_0,\dots,t_m$ such that $t_i =_\alpha^{\nu,\chi,k+1} t_i'$ for each $i$, as well as terms $u,u'$.
  If $\aprel{t_0, [t_1,\dots,t_m], u}$ and $\aprel{t_0', [t_1',\dots,t_m'], u'}$, then $u =_\alpha^{\nu,\chi,k+1} u'$.
\end{enumerate}
\end{lemma}

\begin{proof}
We prove both statements together by a mutual induction on the definition of substitution and application.
Although the proof is not overly complex, it is quite long, and requires multiple applications
of Lemmas \ref{lem:alphafreevar}, \ref{lem:alphaincrease} and \ref{lem:alphaunusedvar}.
    % To prove (\ref{lem:substitutionalpha:subst}), consider the shape of $s$.
    % \begin{itemize}
    % \item Suppose $s = \afun(s_1,\dots,s_n)$.
    %   Then $s' = \afun(s_1',\dots,s_n')$ with $s_i =_\alpha^{\mu,\xi,p+1} s_i'$ for all $i$.
    %   Moreover, $t = \afun(t_1,\dots,t_n)$ with $\subrel{s_i,\gamma,t_i}$ for all $i$,
    %   and $t' = \afun(t_1',\dots,t_n')$ with $\subrel{s_i',\gamma',t_i'}$ for all $i$.
    %   By the induction hypothesis on $\subrel{s_i,\gamma,t_i}$ for each $i$, we have $t_i =_\alpha^{\nu,\chi,k+1} t_i'$ for each $i$.
    %   But then clearly $t =_\alpha^{\nu,\chi,k+1} t'$.
    % \item Suppose $s = x(s_1,\dots,s_n)$.
    %   Then $s' = y(s_1',\dots,s_n')$ with $s_i =_\alpha^{\mu,\xi,p+1} s_i'$ for all $i$, and either $x = y$ and $\mu(x) = \xi(y) = 0$, or $\mu(x) = \xi(y) > 0$.
    %   By definition of $\subrel{s,\gamma,t}$ we have $\aprel{\gamma(x), [t_1,\dots,t_n],t}$ for some $t_1,\dots,t_n$ with $\subrel{s_i,\gamma,t_i}$ for all $i$,
    %   and by definition of $\subrel{s',\gamma',t'}$ we have $\aprel{\gamma'(y), [t_1',\dots,t_n'],t'}$ for some $t_1',\dots,t_n'$ with $\subrel{s_i'\gamma,t_i'}$ for all $i$.
    %   By the induction hypothesis on each $\subrel{s_i,\gamma,t_i}$, we have $t_i =_\alpha^{\nu,\chi,k+1} t_i'$ for all $i$.
    %   In both cases ($x = y$ or not), we have by the assumptions that $\gamma(x) =_\alpha^{\nu,\chi,k+1} \gamma'(y)$.
    %   Hence, by the induction hypothesis on $\aprel{\gamma(x), [t_1,\dots,t_n],t}$, we have $t =_\alpha^{\nu,\chi,k+1} t'$.
    % \item Finally, suppose $s = \abs{x}{u}$. Then:
    %   \begin{itemize}
    %   \item $s' = \abs{x'}{u'}$ for some $u'$ with $u =_\alpha^{\mu[x:=p+1],\xi[x':=p+1],p+2} u'$;
    %   \item $t = \abs{z}{q}$ and $t' = \abs{z'}{q'}$ for some $z,z'$ such that:
    %     \begin{itemize}
    %     \item $z \notin \FV(s) \setminus \domain(\gamma)$ and $z' \notin \FV(s') \setminus \domain(\gamma')$;
    %     \item $z \notin \FV(\gamma(y))$ for any $y \in \domain(\gamma) \cap \FV(s)$ and \\
    %       $z' \notin \FV(\gamma'(y))$ for any $y \in \domain(\gamma') \cap \FV(s')$;
    %     \item $\subrel{u,[x:=z] \cup [y:=\gamma(y) \mid y \in \domain(\gamma) \setminus \{x\},q}$ and \\
    %       $\subrel{u',[x':=z'] \cup [y:=\gamma'(y) \mid y \in \domain(\gamma') \setminus \{x'\},q'}$.
    %     \end{itemize}
    %   \end{itemize}
    %   \ \\
    %   Now, let $\delta := [x:=z] \cup [y:=\gamma(y) \mid w \in \domain(\gamma) \setminus \{ x \}]$ and
    %   $\delta' := [x':=z'] \cup [y:=\gamma'(y) \mid y \in \domain(\gamma') \setminus \{ x' \}]$, so we have
    %   $\subrel{u,\delta,q}$ and $\subrel{u',\delta',q'}$.
    %   Let $\mu_b := \mu[x:=p+1]$ and $\xi_b := \xi[x':=p+1]$ and $\nu_b := \nu[z:=k+1]$ and $\chi_b := \chi[z':=k+1]$.
    %   We apply the induction hypothesis on $\subrel{u,\delta,q}$. (**)
    %   This gives us that $q =_\alpha^{\nu_b,\chi_b,k+2} q'$, and therefore $t = \abs{z}{q} =_\alpha^{\nu,\chi,k+1} \abs{z'}{q'} = t'$ as required.
    % 
    %   (**) To see that we may apply the induction hypothesis to obtain this conclusion, we observe that
    %   $\nu_b,\chi_b$ are functions in $\V \to \{0,\dots,k+1\}$, that $\mu_b,\xi_b$ are functions in $\V \to \{0,\dots,p+1\}$,
    %   and that $u =_\alpha^{\mu_b,\xi_b,p+2} u'$ as observed above.
    %   For the substitutions, we must show that for all $y \in \FV(u)$:
    %   \begin{itemize}
    %   \item if $\mu_b(y) = 0$ then $\delta(y) =_\alpha^{\nu_b,\chi_b,k+2} \delta'(y)$;
    %   \item if $\mu_b(y) > 0$ and $y' \in \FV(u')$ is such that $\mu_b(y) = \xi_b(y')$ then $\delta(y) =_\alpha^{\nu_b,\chi_b,k+2} \delta'(y')$.
    %   \end{itemize}
    %   \ \\
    %   So let $y \in \FV(u)$; if $\mu_b(y) = 0$ then let $y' := y$, otherwise let $y'$ be such that $\mu_b(y) = \xi_b(y')$.
    % 
    %   If $y = x$, then $\mu_b(y) = p + 1$, so we should show the second case.
    %   Since $\xi$ maps to $\{0,\dots,p\}$ and $\xi_b(x') = p + 1$, necessarily $y' = x'$.
    %   Hence we must show: $z = \delta(x) =_\alpha^{\nu_b,\chi_b,k+2} \delta'(x') = z'$.
    %   Since $\nu_b(z) = k + 1 = \chi_b(z')$ this clearly holds.
    % 
    %   Alternatively, if $y \neq x$, then $\mu_b(y) = \mu(y) \leq p$.
    %   \begin{itemize}
    %   \item If $\mu_b(y) > 0$ and $y' \in \FV(u')$ has $\mu_b(y) = \xi_b(y')$, then note that $y' \neq x'$, as $\xi_b(x') = p+1 > \mu_b(y)$.
    %     Hence, $\xi_b(y') = \xi(y')$.
    %   \item Otherwise, $y' = y$ by definition and since $y \in \FV(u) \setminus \{ x \} = \FV(s)$ we can apply Lemma \ref{lem:alphafreevar}
    %     to obtain $y \in \FV(s')$ and $\xi(y) = 0$; so $y \in \FV(u')$ and $y \neq x'$, hence $\xi_b(y) = \xi(y) = 0$.\\
    %   \end{itemize}
    %   Hence, either way, $y' \neq x'$ and $\xi_b(y') = \xi(y')$.
    %   Then also $\delta(y) = \gamma(y)$ and $\delta'(y') = \gamma'(y')$, and by the assumptions on $\gamma,\gamma'$ we have:
    %   $\delta(y) =_\alpha^{\nu,\chi,k+1} \delta'(y)$.
    %   Hence by Lemma \ref{lem:alphaincrease}, $\delta(y) =_\alpha^{\nu,\chi,k+2} \delta'(y)$. (a)
    %   Now, since $y \in \FV(u)$ and $y \neq x$, we have $y \in \FV(s)$.  Similarly, $y' \in \FV(s')$.
    %   By the freshness condition on $z,z'$ we have: $z \notin \FV(\gamma(y)) = \FV(\delta(y))$, and $z' \notin \FV(\delta'(y'))$.
    %   But then by (a) and Lemma \ref{lem:alphaunusedvar}, $\delta(w) =_\alpha^{\nu_b,\chi_b,k+2} \delta'(w')$ as required.
    %   \end{itemize}
    % \ \\
    % To prove (\ref{lem:substitutionalpha:appl}), consider the length $m$.
    % 
    % \begin{itemize}
    % \item If $m = 0$, then $u = t_0 =_\alpha^{\nu,\chi,k+1} t_0' = u'$.
    % \item If $m > 1$, then there exists $q$ such that $\aprel{t_0,[t_1],q}$ and $\aprel{q,[t_2,\dots,t_m],u}$,
    %   and there exists $q'$ such that $\aprel{t_0',[t_1],q'}$ and $\aprel{q',[t_2',\dots,t_m'],u'}$.
    %   By the induction hypothesis, $q =_\alpha^{\nu,\chi,k+1} q'$, so using the induction hypothesis again we complete.
    % \item If $m = 1$ and $t_0 = \afun(s_1,\dots,s_n)$, then $t_0' = \afun(s_1',\dots,s_n')$ with each $s_i =_\alpha^{\nu,\chi,k+1} s_i'$.
    %   Moreover, $u = \afun(s_1,\dots,s_n,t_1)$ and $u' = \afun(s_1',\dots,s_n',t_1')$.
    %   Since $t_1 =_\alpha^{\nu,\chi,k+1} t_1'$ by assumption, indeed $u =_\alpha^{\nu,\chi,k+1} u'$.
    % \item If $m = 1$ and $t_0 = x(u_1,\dots,u_n)$, we complete almost exactly as above.
    % \item Finally, if $m = 1$ and $t_0 = \abs{x}{s}$, then $t_0' = \abs{y}{s'}$ and $s =_\alpha^{\nu[x:=k+1],\chi[y:=k+1],k+2} s'$.
    %   Then $\aprel{t_0,[t_1],u}$ follows from $\subrel{s,[x:=t_1],u}$ and $\aprel{t_0',[t_1'],u'}$ follows from $\subrel{s',[y:=t_1'],u'}$.
    %   Let $\gamma := [x:=t_1]$ and $\gamma' := [y:=t_1']$.
    %   Let $p := k+1$, $\mu := \nu[x:=k+1]$ and $\xi := \chi[y:=k+1]$.
    %   We apply the induction hypothesis on $\subrel{s,\gamma,u}$. (**)
    %   This gives $u =_\alpha^{\nu,\chi,k+1} u'$.
    % 
    %   (**) To see that we may apply the induction hypothesis to obtain this conclusion, we must see that for all $z \in \FV(s)$:
    %   \begin{itemize}
    %   \item if $\mu(z) = 0$ then $\gamma(z) =_\alpha^{\nu,\chi,k+1} \gamma'(z)$;
    %   \item if $\mu(z) > 0$ and $\chi(z') = \mu(z)$ then $\gamma(z) =_\alpha^{\nu,\chi,k+1} \gamma'(z')$.
    %   \\\end{itemize}
    %   We first consider $x = z$.  Then $\mu(z) = k+1$ and the only $z'$ with $\xi(z') = \mu(z)$ is $y$.
    %   Since $\gamma(x) = t_1 =_\alpha^{\nu,\chi,k+1} t_1' = \gamma'(y')$ by assumption, this case is done.
    %   In all other cases, $\mu(z) = \nu(z)$ and $\gamma(z) = z$. We consider both cases.
    %   \begin{itemize}
    %   \item If $\mu(z) = 0$, then note that since $z \in \FV(s) \setminus \{x\}$, we have $z \in \FV(t_0)$.
    %     By Lemma \ref{lem:alphafreevar}, also $z \in \FV(t_0')$ and $\chi(z) = 0$.  As $z \in \FV(t_0') =
    %     \FV(s') \setminus \{ y\}$, we have $z \neq y$, so $\gamma'(z) = z$.
    %     Then $\gamma(z) = z =_\alpha^{\nu,\chi,k+1} z = \gamma'(z)$ indeed holds.
    %   \item If $\mu(z) > 0$ and $\xi(z') > 0$ then because $z \neq x$ we have $\mu(z) \leq k$ and $\xi(z') \leq k$;
    %     hence $z \neq x$ and $z' \neq y$.  Hence, $\gamma(z) = z =_\alpha^{\nu,\chi,k+1} z' = \gamma'(z')$ indeed holds
    %     (as $\nu(z) = \mu(z) = \xi(z') = \chi(z')$).
    %     \qedhere
    %   \end{itemize}
    % \end{itemize}
\end{proof}

\begin{corollary}\label{cor:substitutionalpha} We have the following results:
\begin{enumerate}
\item If $s =_\alpha s'$ and $\subrel{s,\gamma,t}$ and $\subrel{s',\gamma',t'}$ and
  $\gamma(x) =_\alpha \gamma'(x)$ for all $x \in \FV(s)$, then $t =_\alpha t'$.
\item If $s_0 =_\alpha s_0', s_1 =_\alpha s_1', \dots, s_m =_\alpha s_m'$ and $\aprel{s_0, [s_1,\dots,s_m], t}$
  and $\aprel{s_0',[s_1',\dots,s_m'],t'}$ then $t =_\alpha t'$.
\end{enumerate}
\end{corollary}

\subsection{Renamings}

Renamings are convenient to reason with, since the definition of substitution does not require the
mutual substitution with the definition of application when only renamings are involved:
$\subrel{x(s_1,\dots,s_n),\gamma,t}$ if and only if (1) $x \notin \domain(\gamma)$ and $t =
x(t_1,\dots,t_n)$ with $\subrel{s_i,\gamma,t_i}$ for all $i$; or (2) $\gamma(x) = y$ and $t =
y(t_1,\dots,t_n)$ with $\subrel{s_i,\gamma,t_i}$ for all $i$.
We list some important helper results.

The first, technical result, shows a relation between renaming an $\alpha$-equivalence.  This will
be very useful towards Lemma \ref{lem:alphasubst}, where we will demonstrate that
$\alpha$-equivalence can be defined in terms of renamings.

\begin{lemma}\label{lem:alphasubst:basis}
Let $\gamma = [x_1:=y_1,\dots,x_n:=y_n]$ be a renaming, where $x_1,\dots,x_n$ are all pairwise
distinct, and $y_1,\dots,y_n$ are all pairwise distinct.
Let $s$ be a term such that $\{y_1,\dots,y_n\} \cap \FV(s) \subseteq \{x_1,\dots,x_n\}$.
Let $\mu,\xi$ be a function from $\Vbound$ to $\{0,\dots,k\}$ such that $\mu(x_i) = \xi(y_i) > 0$
for all $i \in \{1,\dots,n\}$, and $\mu(z) = 0$ for all $z \notin \{x_1,\dots,x_n\}$ and $\xi(z) =
0$ for all $z \notin \{y_1,\dots,y_n\}$.
Then there exists $t$ such that $s =_\alpha^{\mu,\xi,k+1} t$ and $\subrel{s,\gamma,t}$.
\end{lemma}

\begin{proof}
By induction on the size of $s$.

\begin{itemize}
\item
  If $z = \afun(s_1,\dots,s_m)$ then by the induction hypothesis there exist $t_1,\dots,t_m$ such
  that $s_i =_\alpha^{\mu,\xi,k+1} t_i$ and $\subrel{s_i,\gamma,t_i}$ (we can apply the induction
  hypothesis because $\FV(s_i) \subseteq \FV(s)$).  From this, we immediately obtain both that
  $s =_\alpha^{\mu,\xi,k+1} t$ and that $\subrel{s,\gamma,t}$.
\item
  If $s = z(s_1,\dots,s_m)$ with $z \notin \{x_1,\dots,x_n\}$, then by the induction hypothesis
  there exist $t_1,\dots,t_m$ such that $s_i =_\alpha^{\mu,\xi,k+1} t_i$ and $\subrel{s_i,\gamma,
  t_i}$ for all $i$; we let $t := z(t_1,\dots,t_m)$.  Then $\subrel{s,\gamma,t}$ follows
  immediately (as $z \notin \domain(\gamma)$).  Also $s =_\alpha^{\mu,\xi,k+1} t$: $\mu(z) = 0$
  (since $z \notin \{x_1,\dots,x_n\}$ by assumption), and $\xi(z) = 0$ because $z \notin \{y_1,
  \dots,y_n\}$: $z \in \FV(s)$, so if some $y_i = z$ then $z \in \{x_1,\dots,x_n\}$ by assumption;
  contradiction.
\item
  If $s = x_i(s_1,\dots,s_m)$ then by the induction hypothesis there exist $t_1,\dots,t_m$ such
  that $s_i =_\alpha^{\mu,\xi,k+1} t_i$ and $\subrel{s_i,\gamma,t_i}$ for all $i$; we let
  $t := y_i(t_1,\dots,t_m)$.  Then $s =_\alpha^{\mu,\xi,k+1} t$ because $\mu(x_i) = \xi(y_i) > 0$,
  and $\subrel{s,\gamma,t}$ because $\aprel{y_i,[t_1,\dots,t_m],y_i(t_1,\dots,t_m)}$ clearly holds.
\item 
  If $s = \abs{x_{n+1}}{s'}$ and $x_{n+1}$ is distinct from all previous $x_i$ then let
  $y_{n+1}$ be a variable of the same type, distinct from all variables in $\FV(s)$ and from
  $y_1,\dots,y_n$.
  Then:
  \begin{itemize}
  \item[] $\{y_1,\dots,y_{n+1}\} \cap \FV(s')$
  \item[] $\subseteq \{ y_1,\dots,y_{n+1}\} \cap (\FV(s) \cup \{x_{n+1}\})$
  \item[] $\subseteq (\{ y_1,\dots,y_{n+1}\} \cap \FV(s)) \cup \{ x_{n+1} \}$
  \item[] $= (\{ y_1,\dots,y_n \} \cap \FV(s)) \cup \{x_{n+1}\}$ because $y_{n+1} \notin \FV(s)$ by
    assumption,
  \item[] $\subseteq \{ x_1,\dots,x_n \} \cup \{ x_{n+1} \} = \{ x_1,\dots,x_{n+1} \}$. \\
  \end{itemize}
  Let $\mu' = \mu[x_{n+1}:=k+1]$ and $\xi' = \xi[y_{n+1}:=k+1]$.
  Then $\mu'(z) = 0$ for $z \notin \{x_1,\dots,x_{n+1}\}$ and $\xi'(z) = 0$ for $z \notin \{y_1,
  \dots,y_{n+1}\}$; clearly $\mu'(x_i) = \xi'(x_i)$ for all $i$ (since either $i = n+1$, in which
  case both sides are $k+1$, or $i \leq n$, in which case $x_i \neq x_{n+1}$ by assumption and
  $y_i \neq y_{n+1}$ by choice of $y_{n+1}$).
  Hence, we can aply the induction hypothesis to obtain $t'$ such that
  $s' =_\alpha^{\mu',\xi',k+1} t'$ and $\subrel{s',\gamma \cup [x_{n+1}:=y_{n+1}],t'}$.
  Since $\gamma \cup [x_{n+1}:=y_{n+1}]$ is exactly $[x_{n+1}:=y_{n+1}] \cup [z:=\gamma(z) \mid 
  z \in \domain(\gamma) \wedge z \neq x_{n+1}]$, and because $y_{n+1}$ is not in $\FV(s)$ or in
  $\FV(\gamma(z))$ for any $z \in \domain(\gamma) = \{x_1,\dots,x_n\}$, 
  we obtain $\subrel{s,\gamma,t}$.
\item
  If $s = \abs{x_i}{s'}$, then $x_i \notin \FV(s)$.  Then:
  \begin{itemize}
  \item[] $\{y_1,\dots,y_n\} \cap \FV(s')$
  \item[] $\subseteq \{ y_1,\dots,y_n\} \cap (\FV(s) \cup \{x_i\})$
  \item[] $\subseteq (\{ y_1,\dots,y_n \} \cap \FV(s)) \cup \{x_i\}$
  \item[] $\subseteq \{x_1,\dots,x_n\} \cup \{x_i\} = \{x_1,\dots,x_n\}$. \\
  \end{itemize}
  Let $\mu' := \mu[x_i:=k+1]$ and $\xi' := \xi[y_i:=k+1]$.  Then $\mu'(z) = 0$ for all $z \notin
  \{x_1,\dots,x_n\}$ and $\xi'(z) = 0$ for all $z \notin \{y_1,\dots,y_n\}$, $\mu'(x_j) = \xi'(y_j)
  > 0$ for all $j \in \{1,\dots,n\}$, and $\mu',\xi'$ map $\Vbound$ to $\{0,\dots,k+2\}$.
  Hence by the induction hypothesis there is $t'$ such that $s' =_\alpha^{\mu',\xi',k+1} t'$ and
  $\subrel{s',\gamma,t'}$.
  Let $t = \abs{y_i}{t'}$.  Then clearly $s =_\alpha^{\mu,\xi,k+1} t$.  We also have $\subrel{s,
  \gamma,t}$, because:
  (a) $\gamma = [x_i:=y_i] \cup [z:=\gamma(z) \mid z \in \domain(\gamma) \setminus \{x_i\}]$;
  (b) $\{y_i\} \cap \FV(s) \subseteq \{x_1,\dots,x_n\}$, so $y_i \notin \FV(s) \setminus
    \domain(\gamma)$;
  (c) $y_i$ is unequal to all other $y_j$ and $x_i \notin \FV(s)$, so $y_i \neq \gamma(x_j)$ for
    any $x_j \in \FV(s) \cap \domain(\gamma)$.
  \qedhere
\end{itemize}
\end{proof}

Another useful technical result is the following, which states that the free variables under a
renaming are exactly the renaming of the free variables:

\begin{lemma}\label{lem:freerename}
Let $\gamma$ be a renaming, and $s^\gamma$ such that $\subrel{s,\gamma,s^\gamma}$.
Then $\FV(s^\gamma) = (\FV(s) \setminus \domain(\gamma)) \cup
\{ \gamma(x) \mid x \in \FV(s) \cap \domain(\gamma) \}$.
\end{lemma}

\begin{proof}
By induction on the form of $s$.

If $s = \afun(s_1,\dots,s_n)$ then $\FV(s) = \bigcup_{i=1}^n \FV(s_i)$ and $s^\gamma =
\afun(s_1^\gamma,\dots,s_n^\gamma)$ for some $s_1^\gamma,\dots,s_n^\gamma$ with
$\subrel{s_i,\gamma,s_i^\gamma}$ for all $i$.
Now,
\[
\begin{array}{cl}
& \FV(s^\gamma) \\
= & \bigcup_{i=1}^n. \FV(s_i^\gamma) \\
= & \bigcup_{i=1}^n ((\FV(s_i) \setminus \domain(\gamma)) \cup \{ \gamma(x) \mid x \in \FV(s_i)
  \cap \domain(\gamma) \})\ \text{(by the induction hypothesis)} \\
= & (\bigcup_{i=1}^n (\FV(s_i) \setminus \domain(\gamma))) \cup
    (\bigcup_{i=1}^n \{ \gamma(x) \mid x \in \FV(s_i) \cap \domain(\gamma) \}) \\
= & ((\bigcup_{i=1}^n \FV(s_i)) \setminus \domain(\gamma)) \cup
    \{ \gamma(x) \mid x \in (\bigcup_{i=1}^n \FV(s_i)) \cap \domain(\gamma) \} \\
= & (\FV(s) \setminus \domain(\gamma)) \cup
    \{ \gamma(x) \mid x \in \FV(s) \cap \domain(\gamma) \} \\
\end{array}
\]
If $s = z(s_1,\dots,s_n)$ with $z \notin \domain(\gamma)$ then $s^\gamma = z(s_1^\gamma,\dots,
s_n^\gamma)$ for suitable $s_1^\gamma,\dots,s_n^\gamma$, and (a) $\FV(s) \setminus \domain(\gamma)
= \{ z \} \cup \bigcup_{i = 1}^n (\FV(s_i) \setminus \domain(\gamma))$ and (b) $\FV(s) \cap
\domain(\gamma) = \bigcup_{i=1}^n (\FV(s_i) \cap \domain(\gamma))$.  Hence:
\[
\begin{array}{cl}
& \FV(s^\gamma) \\
= & \{ z \} \cup \bigcup_{i=1}^n. \FV(s_i^\gamma) \\
= & \{ z \} \cup ((\bigcup_{i=1}^n \FV(s_i)) \setminus \domain(\gamma)) \cup
  \{ \gamma(x) \mid x \in (\bigcup_{i=1}^n \FV(s_i)) \cap \domain(\gamma) \}\ 
  \text{(as above)} \\
= & (\FV(s) \setminus \domain(\gamma)) \cup \{ \gamma(x) \mid x \in (\bigcup_{i=1}^n \FV(s_i)) \cap \domain(\gamma) \}\ 
  \text{(by (a))} \\
= & (\FV(s) \setminus \domain(\gamma)) \cup \{ \gamma(x) \mid x \in \FV(s) \cap \domain(\gamma) \}\ 
  \text{(by (b))} \\
\end{array}
\]
If $s = z(s_1,\dots,s_n)$ with $z \in \domain(\gamma)$ and $\gamma(z) = z'$ then $s^\gamma =
z'(s_1^\gamma,\dots,s_n^\gamma)$ for suitable $s_1^\gamma,\dots,s_n^\gamma)$, and
(c) $\FV(s) \setminus \domain(\gamma) = \bigcup_{i=1}^n (\FV(s_i) \setminus \domain(\gamma))$
and (d) $\FV(s) \cap \domain(\gamma) = \{ z \} \cup \bigcup_{i=1}^n (\FV(s_i) \cap
\domain(\gamma))$.  Hence:
\[
\begin{array}{cl}
& \FV(s^\gamma) \\
= & \{ z' \} \cup \bigcup_{i=1}^n. \FV(s_i^\gamma) \\
= & \{ z' \} \cup ((\bigcup_{i=1}^n \FV(s_i)) \setminus \domain(\gamma)) \cup
  \{ \gamma(x) \mid x \in (\bigcup_{i=1}^n \FV(s_i)) \cap \domain(\gamma) \}\ 
  \text{(as before)} \\
= & (\FV(s) \setminus \domain(\gamma)) \cup \{ z' \} \cup \{ \gamma(x) \mid x \in (\bigcup_{i=1}^n \FV(s_i)) \cap \domain(\gamma) \}\ 
  \text{(by (c))} \\
= & (\FV(s) \setminus \domain(\gamma)) \cup \{ \gamma(z) \} \cup \{ \gamma(x) \mid x \in (\bigcup_{i=1}^n \FV(s_i)) \cap \domain(\gamma) \} \\
= & (\FV(s) \setminus \domain(\gamma)) \cup \{ \gamma(x) \mid x \in (\{ z \} \cup \bigcup_{i=1}^n \FV(s_i)) \cap \domain(\gamma) \} \\
= & (\FV(s) \setminus \domain(\gamma)) \cup \{ \gamma(x) \mid x \in \FV(s) \cap \domain(\gamma) \}\ \text{(by (d))} \\
\end{array}
\]

If $s = \abs{z}{t}$ then $s^\gamma = \abs{z'}{t^\delta}$ where $\delta = [z:=z'] \cup [x:=
\gamma(x) \mid x \in \domain(\gamma) \setminus \{z\}]$, and $\subrel{t,\delta,t^\delta}$.  Note
that $\delta$ is still a renaming, that (e) $z' \notin \FV(s) \setminus \domain(\gamma)$, and
(f) $z' \neq \gamma(x)$ for any $x \in \FV(s) \cap \domain(\gamma)$.
Hence:
\[
\begin{array}{cl}
& \FV(s^\gamma) \\
= & \FV(t^\delta) \setminus \{ z' \} \\
= & ((\FV(t) \setminus \domain(\delta)) \cup \{ \delta(x) \mid x \in \FV(t) \cap \domain(\delta) \}) \setminus \{z'\}\ 
  \text{(by the induction hypothesis)} \\
= & ((\FV(t) \setminus \domain(\delta)) \setminus \{z'\}) \cup
  (\{ \delta(x) \mid x \in \FV(t) \cap \domain(\delta) \} \setminus \{z'\}) \\
= & ((\FV(t) \setminus (\domain(\gamma) \cup \{z\})) \setminus \{z'\})\ \cup \\ &
  (\{ \delta(x) \mid x \in (\FV(t) \cap (\domain(\gamma) \setminus \{z\})) \cup (\FV(t) \cap \{z\}) \} \setminus \{z'\}) \\
= & (((\FV(t) \setminus \{z\}) \setminus \domain(\gamma)) \setminus \{z'\})\ \cup \\ &
  (\{ \delta(x) \mid x \in \FV(t) \cap (\domain(\gamma) \setminus \{z\}) \} \setminus \{z'\}) \cup
  (\{ \delta(x) \mid x \in \FV(t) \cap \{z\} \} \setminus \{z'\}) \\
= & ((\FV(s) \setminus \domain(\gamma)) \setminus \{z'\})\ \cup \\ &
  (\{ \delta(x) \mid x \in \FV(t) \cap (\domain(\gamma) \setminus \{z\}) \} \setminus \{z'\})\ 
  (\text{as}\ \delta(x)\ \text{for}\ x \in \{z\}\ \text{can only be}\ z') \\
= & (\FV(s) \setminus \domain(\gamma)) \cup
  (\{ \delta(x) \mid x \in \FV(t) \cap (\domain(\gamma) \setminus \{z\}) \} \setminus \{z'\})\ 
  \text{(by (e))} \\
= & (\FV(s) \setminus \domain(\gamma)) \cup
  (\{ \gamma(x) \mid x \in \FV(t) \cap (\domain(\gamma) \setminus \{z\}) \} \setminus \{z'\}) \\
= & (\FV(s) \setminus \domain(\gamma)) \cup
  (\{ \gamma(x) \mid x \in \FV(s) \cap \domain(\gamma) \} \setminus \{z'\}) \\
= & (\FV(s) \setminus \domain(\gamma)) \cup
  (\{ \gamma(x) \mid x \in \FV(s) \cap \domain(\gamma) \}\ \text{(by (f))} \\
\end{array}
\]
\end{proof}

====================



This technical result easily allows us to obtain the following practical result:

\begin{lemma}\label{lem:alphasubst}
$\abs{x}{s} =_\alpha \abs{y}{t}$ if and only if $s[x:=y] =_\alpha t$.
\end{lemma}

\begin{proof}
We have $\abs{x}{s} =_\alpha \abs{y}{t}$ if and only if
$\abs{x}{s} =_\alpha^{\mu_0,\xi_0,1} \abs{y}{t}$ if and only if
$s =_\alpha^{\mu_0[x:=1],\xi[y:=1],2} t$.

Now, if $y \in \FV(s) \setminus \{x\}$ TODO: $\FV(s\gamma) \subseteq \FV(s)\setminus
\domain(\gamma) \cup \bigcup_{y \in \FV(s)} \FV(\gamma(y))$
\end{proof}

\subsection{Substitution and $\alpha$-equivalence properties}


Ook TODO: $\abs{x}{s} =_\alpha \abs{z}{t}$ if and only if $s =_\alpha t[z:=x]$

\begin{lemma}\label{lem:unusedsubstitution}
Let $\gamma = \delta \cup \eta$, and let $s$ be a term with $\FV(s) \cap \domain(\eta) = \emptyset$.
Then for all $t$: $\subrel{s,\gamma,t}$ if and only if $\subrel{s,\delta,t}$.
\end{lemma}

\begin{proof}
By induction on the size of $s$.
All cases are straightforward.
    % \begin{itemize}
    % \item If $s = \afun(s_1,\dots,s_n)$ then note that $\FV(s_i) \subseteq \FV(s)$ for all $i$, so we
    %   can apply the induction hypothesis to obtain that for all $i$ and $t_i$: $\subrel{s_i,\gamma,t_i}$
    %   if and only if $\subrel{s_i,\delta,t_i}$. Hence, $\subrel{s,\gamma,t}$ holds if and only if
    %   $t = \afun(t_1,\dots,t_n)$ with $\subrel{s_i,\gamma,t_i}$ for all $i$, if and only if
    %   $t = \afun(t_1,\dots,t_n)$ with $\subrel{s_i,\delta,t_i}$ for all $i$, if and only if
    %   $\subrel{s,\delta,t}$.
    % \item If $s = x(s_1,\dots,s_n)$ and $x \notin \domain(\gamma)$, then also $x \notin \domain(\delta)$
    %   so we complete as above.
    % \item If $s = x(s_1,\dots,s_n)$ and $x \in \domain(\gamma)$, then note that $x \in \FV(s)$, so $x
    %   \notin \domain(\eta)$, so necessarily $x \in \domain(\delta)$. Then
    %   $\subrel{s,\gamma,t}$ if and only if $\aprel{\delta(x),[t_1,\dots,t_n],t}$ for some $t_1,\dots,
    %   t_n$ such that $\subrel{s_i,\gamma,t_i}$ for all $i$; and
    %   $\subrel{s,\delta,t}$ if and only if $\aprel{\delta(x),[t_1,\dots,t_n],t}$ for some $t_1,\dots,
    %   t_n$ such that $\subrel{s_i,\delta,t_i}$ for all $i$.
    %   As $\subrel{s_i,\gamma,t_i}$ if and only if $\subrel{s_i,\delta,t_i}$ by the induction hypothesis,
    %   we are done.
    % \item If $s = \abs{x}{s'}$ then for any variable $z$ denote $\gamma^{x\mapsto z}$ for the
    %   substitution $[x:=z] \cup [y:=\gamma(y) \mid y \in \domain(\gamma) \setminus \{x\}]$ and similar
    %   for $\delta^{x\mapsto z}$.
    % 
    %   Now, $\subrel{s,\gamma,t}$ if and only if $t = \abs{z}{t'}$ for some $t'$ with
    %   $\subrel{s',\gamma^{x\mapsto z},t'}$ and $z \notin \FV(s) \setminus \domain(\gamma)$ and $z \notin
    %   \FV(\gamma(y))$ for any $y \in \domain(\gamma) \cap \FV(s)$.
    %   But then also:
    %   \begin{itemize}
    %   \item $z \notin \FV(s) \setminus \domain(\delta)$: if $z \in \FV(s)$ but not $z \in \domain(\delta)$,
    %     then since $z \cap \domain(\eta) = \emptyset$ also $z \notin \domain(\gamma)$; this contradicts
    %     $z \notin \FV(s) \setminus \domain(\gamma)$;
    %   \item $z \notin \domain(\delta) \cap \FV(\delta(y))$ for any $y \in \FV(s)$ since
    %     $\domain(\gamma) \cap \FV(s) = \domain(\delta) \cap \FV(s)$ and $\gamma(y) = \delta(y)$ on this
    %     domain;
    %   \item $\subrel{s',\delta^{x\mapsto z},t'}$: this holds by the induction hypothesis for
    %     $\gamma^{x\mapsto z} = \delta^{x\mapsto z} \cup [y:=\eta(y) \mid y \in \domain(\eta) \setminus
    %     \{x\}]$.  After all, clearly $\FV(s') \cap [y:=\eta(y) \mid y \in \domain(\eta) \setminus \{x
    %     \}] \subseteq (\FV(s) \cup \{ x \}) \cap (\domain(\eta) \setminus \{ x \}) = \FV(s) \setminus
    %     \domain(\eta)$.
    %   \end{itemize}
    %   \qedhere
    % \end{itemize}
\end{proof}


\end{document}

%\section{Unconstrained first-order term rewriting}
%
%Although first-order (many-sorted) term rewriting systems can be seen as a kind of higher-order
%term rewriting system, we will present their definition separately first, and later explain how
%they can be viewed as part of the larger framework.
%
%\subsection{Terms} When considering \emph{first-order} term rewriting, we limit interest to $\F$
%with the following property: for every $(\afun : \atype) \in \F$ we have $\order(\atype) \leq 1$.
%First-order terms are those expressions $s$ such that $s : \asort$ can be derived for some
%\emph{base type} $\asort$ using the following clauses:
%\begin{itemize}
%\item if $(\afun : \atype_1 \arrtype \dots \arrtype \atype_n \arrtype \asort) \in \F$ and
%  $s_1 : \atype_1,\dots,s_n : \atype_n$ then $\afun(s_1,\dots,s_n) : \asort$;
%\item if $(\avar : \asort) \in \V$ then $\avar : \asort$.
%\end{itemize}
%We denote $\FOTerms(\F,\V)$ for the set of all first-order terms $s$.
%A first-order term of the form $\afun(s_1,\dots,s_n)$ is called a \emph{functional term} and
%$\afun$ is its root; a term $\avar$ is simply called a variable.
%If $s : \asort$ then we say that $\asort$ is the type of $s$; it is clear from the definitions
%above that each term has a unique type (which is a base type).
%
%The set $\FV(s)$ of \emph{variables} of a term $s$ is inductively defined as follows:
%\begin{itemize}
%\item $\FV(\afun(s_1,\dots,s_n)) = \FV(s_1) \cup \dots \cup \FV(s_n)$;
%\item $\FV(\avar) = \{ \avar \}$.
%\end{itemize}
%%That is, $\FV(s)$ contains all variables in $s$.
%
%\bigskip
%The \emph{subterm} relation $\subtermeq$ is defined as follows:
%\begin{itemize}
%\item $s \subtermeq s$ for all $s$;
%\item $s \subtermeq \afun(s_1,\dots,s_n)$ if $s \subtermeq s_i$ for some $i$.
%\end{itemize}
%If $s \subtermeq t$ we say that $s$ \emph{is a subterm of} $t$.
%
%\subsection{Substitution}
%
%A substitution is a function $\gamma$ that maps each variable $\avar \in \V$ to a term
%$\gamma(\avar)$ of the same type.  A substitution is applied to an arbitrary first-order term as
%follows:
%\begin{itemize}
%\item $\afun(s_1,\dots,s_n)\gamma = \afun(s_1\gamma,\dots,s_n\gamma)$;
%\item $\avar\gamma = \gamma(\avar)$.
%\end{itemize}
%
%The \emph{domain} $\domain(\gamma)$ of a substitution $\gamma$ is the set of all variables $x$
%such that $\gamma(x) \neq x$.
%We denote $[x_1:=s_1,\dots,x_n:=s_n]$ for the substitution $\gamma$ with $\gamma(x_i) = s_i$ for
%$1 \leq i \leq n$ and $\gamma(y) = y$ for $y \notin \{x_1,\dots,x_n\}$.
%For two substitutions $\gamma$ and $\delta$, we let $\gamma\delta$ denote the substitution that
%maps each variable $x$ to $\gamma(x)\delta$.
%
%\subsection{Positions}
%
%The \emph{positions} of a given first-order term are the paths to specific subterms, defined as
%follows:
%
%\begin{itemize}
%\item $\Positions(\afun(s_1,\dots,s_n)) = \{ \epsilon \} \cup \{ i \cdot p \mid 1 \leq i \leq n
%  \wedge p \in \Positions(s_i) \}$;
%\item $\Positions(\avar) = \{ \epsilon \}$.
%\end{itemize}
%Note that positions are associated to a term; thus, not every integer sequence is a position.
%
%For a term $s$ and a position $p \in \Positions(s)$, the \emph{subterm of $s$ at position $p$},
%denoted $s|_p$, is defined as follows:
%\begin{itemize}
%\item $s|_\epsilon = s$;
%\item $\afun(s_1,\dots,s_n)|_{i \cdot p} = s_i|_p$;
%\end{itemize}
%
%Note that $t \subtermeq s$ if and only if there is some position $p \in \Positions(s)$ with
%$t = s|_p$.
%If $s|_p$ has the same type as some term $t$, then $s[t]_p$ denotes $s$ with the subterm at position
%$p$ replaced by $t$.  Formally, $s[t]_p$ is obtained as follows:
%\begin{itemize}
%\item $s[t]_p = t$;
%\item $\afun(s_1,\dots,s_n)[t]_{i \cdot p} = \afun(s_1,\dots,s_{i-1},s_i[t]_p,s_{i+1},\dots,s_n)$.
%\end{itemize}
%
%\subsection{Rules and reduction}
%
%A rule is a pair $\ell \arrz r$ of two terms with the same type.
%For a given set of rules $\Rules$, the reduction relation $\arr{\Rules}$ is given by:
%\begin{itemize}
%\item if there exist $\ell \arrz r \in \Rules$ and $p \in \Positions(s)$ and substitution $\gamma$
%  such that $s|_p = \ell\gamma$, then $s \arr{\Rules} s[r\gamma]_p$.
%\end{itemize}
%
%\bigskip
%A \emph{first-order term rewriting system (TRS)} is an abstract rewriting system of the form
%$(\FOTerms(\F,\V),\arr{\Rules})$.
%
%In principle, we have defined a \emph{many-sorted} TRS here; a traditional unsorted TRS is obtained
%by limiting interest to the case $\Sorts = \{ \unitsort \}$.


